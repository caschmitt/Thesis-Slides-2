\documentclass[12pt]{beamer}
\usepackage{amsmath}
\usepackage{amssymb,tabu }

\setbeamertemplate{footline}{\hfill\insertframenumber/\inserttotalframenumber\ }

\newcommand{\bbC}{\mathbb{C}}
\newcommand{\bbF}{\mathbb{F}}
\newcommand{\bbH}{\mathbb{H}}
\newcommand{\bbK}{\mathbb{K}}
\newcommand{\bbR}{\mathbb{R}}
\newcommand{\bbT}{\mathbb{T}}
\newcommand{\bbZ}{\mathbb{Z}}

\newcommand{\bfA}{\mathbf{A}}
\newcommand{\bfb}{\mathbf{b}}
\newcommand{\bfB}{\mathbf{B}}
\newcommand{\bfC}{\mathbf{C}}
\newcommand{\bfe}{\mathbf{e}}
\newcommand{\bfE}{\mathbf{E}}
\newcommand{\bfF}{\mathbf{F}}
\newcommand{\bfg}{\mathbf{g}}
\newcommand{\bfG}{\mathbf{G}}
\newcommand{\bfH}{\mathbf{H}}
\newcommand{\bfI}{\mathbf{I}}
\newcommand{\bfJ}{\mathbf{J}}
\newcommand{\bfM}{\mathbf{M}}
\newcommand{\bfP}{\mathbf{P}}
\newcommand{\bfQ}{\mathbf{Q}}
\newcommand{\bfS}{\mathbf{S}}
\newcommand{\bfU}{\mathbf{U}}
\newcommand{\bfV}{\mathbf{V}}
\newcommand{\bfW}{\mathbf{W}}
\newcommand{\bfx}{\mathbf{x}}
\newcommand{\bfX}{\mathbf{X}}
\newcommand{\bfy}{\mathbf{y}}
\newcommand{\bfz}{\mathbf{z}}
\newcommand{\bfZ}{\mathbf{Z}}

\newcommand{\bfc}{\mathbf{c}}
\newcommand{\bff}{\mathbf{f}}


\newcommand{\bfone}{\boldsymbol{1}}
\newcommand{\bfzero}{\boldsymbol{0}}

\newcommand{\bfdelta}{\boldsymbol{\delta}}
\newcommand{\bfphi}{\boldsymbol{\varphi}}
\newcommand{\bfPi}{\boldsymbol{\Pi}}
\newcommand{\bfpsi}{\boldsymbol{\psi}}
\newcommand{\bfPhi}{\boldsymbol{\Phi}}
\newcommand{\bfPsi}{\boldsymbol{\Psi}}
\newcommand{\bfSigma}{\boldsymbol{\Sigma}}

\newcommand{\calB}{\mathcal{B}}
\newcommand{\calD}{\mathcal{D}}
\newcommand{\calG}{\mathcal{G}}
\newcommand{\calH}{\mathcal{H}}
\newcommand{\calK}{\mathcal{K}}
\newcommand{\calM}{\mathcal{M}}
\newcommand{\calN}{\mathcal{N}}
\newcommand{\calR}{\mathcal{R}}
\newcommand{\calU}{\mathcal{U}}
\newcommand{\calV}{\mathcal{V}}

\newcommand{\rmc}{\mathrm{c}}
\newcommand{\rmC}{\mathrm{C}}
\newcommand{\rmi}{\mathrm{i}}
\newcommand{\rmT}{\mathrm{T}}

\newcommand{\dist}{\operatorname{dist}}
\newcommand{\rank}{\operatorname{rank}}
\newcommand{\Span}{\operatorname{span}}
\newcommand{\Tr}{\operatorname{Tr}}
\newcommand{\ETF}{{\operatorname{ETF}}}
\newcommand{\QSD}{{\operatorname{QSD}}}
\newcommand{\SRG}{{\operatorname{SRG}}}
\newcommand{\BIBD}{{\operatorname{BIBD}}}
\newcommand{\RBIBD}{{\operatorname{RBIBD}}}
\newcommand{\Fro}{\operatorname{Fro}}

\newcommand{\coh}{\operatorname{coh}}
\newcommand{\spark}{\operatorname{spark}}

\newcommand{\abs}[1]{|{#1}|}
\newcommand{\bigabs}[1]{\bigl|{#1}\bigr|}
\newcommand{\Bigabs}[1]{\Bigl|{#1}\Bigr|}
\newcommand{\biggabs}[1]{\biggl|{#1}\biggr|}
\newcommand{\Biggabs}[1]{\Biggl|{#1}\Biggr|}

\newcommand{\paren}[1]{({#1})}
\newcommand{\bigparen}[1]{\bigl({#1}\bigr)}
\newcommand{\Bigparen}[1]{\Bigl({#1}\Bigr)}
\newcommand{\biggparen}[1]{\biggl({#1}\biggr)}
\newcommand{\Biggparen}[1]{\Biggl({#1}\Biggr)}

\newcommand{\bracket}[1]{[{#1}]}
\newcommand{\bigbracket}[1]{\bigl[{#1}\bigr]}
\newcommand{\Bigbracket}[1]{\Bigl[{#1}\Bigr]}
\newcommand{\biggbracket}[1]{\biggl[{#1}\biggr]}
\newcommand{\Biggbracket}[1]{\Biggl[{#1}\Biggr]}

\newcommand{\set}[1]{\{{#1}\}}
\newcommand{\bigset}[1]{\bigl\{{#1}\bigr\}}
\newcommand{\Bigset}[1]{\Bigl\{{#1}\Bigr\}}
\newcommand{\biggset}[1]{\biggl\{{#1}\biggr\}}
\newcommand{\Biggset}[1]{\Biggl\{{#1}\Biggr\}}

\newcommand{\norm}[1]{\|{#1}\|}
\newcommand{\bignorm}[1]{\bigl\|{#1}\bigr\|}
\newcommand{\Bignorm}[1]{\Bigl\|{#1}\Bigr\|}
\newcommand{\biggnorm}[1]{\biggl\|{#1}\biggr\|}
\newcommand{\Biggnorm}[1]{\Biggl\|{#1}\Biggr\|}

\newcommand{\ip}[2]{\langle{#1},{#2}\rangle}
\newcommand{\bigip}[2]{\bigl\langle{#1},{#2}\bigr\rangle}
\newcommand{\Bigip}[2]{\Bigl\langle{#1},{#2}\Bigr\rangle}
\newcommand{\biggip}[2]{\biggl\langle{#1},{#2}\biggr\rangle}
\newcommand{\Biggip}[2]{\Biggl\langle{#1},{#2}\Biggr\rangle}

\newcommand{\alphi}{\renewcommand{\labelenumi}{(\alph{enumi})}}
\newcommand{\alphii}{\renewcommand{\labelenumii}{(\alph{enumii})}}
\newcommand{\alphiii}{\renewcommand{\labelenumiii}{(\alph{enumiii})}}
\newcommand{\romani}{\renewcommand{\labelenumi}{(\roman{enumi})}}
\newcommand{\romanii}{\renewcommand{\labelenumii}{(\roman{enumii})}}
\newcommand{\romaniii}{\renewcommand{\labelenumiii}{(\roman{enumiii})}}
\newcommand{\arabici}{\renewcommand{\labelenumi}{(\arabic{enumi})}}
\newcommand{\arabicii}{\renewcommand{\labelenumii}{(\arabic{enumii})}}
\newcommand{\arabiciii}{\renewcommand{\labelenumiii}{(\arabic{enumiii})}}

\setlength{\labelwidth}{0pt}
\setbeamertemplate{navigation symbols}{}

\setbeamercolor*{item}{fg=black}

\title{Harmonic Equiangular Tight Frames\\Comprised of Regular Simplices}
\author{%
2d Lt Courtney Schmitt}
\institute{
Dr. Matthew Fickus - Advisor\\
Dr. Dursun Bulutoglu - Committee Member\\
Dr. Mark Oxley - Committee Member}

\date{February 21, 2019}
\titlegraphic{\tiny{The views expressed in this talk are those of the speaker and do not reflect the official policy\\ or position of the United States Air Force, Department of Defense, or the U.S. Government.}}

%Abstract: An equiangular tight frame (ETF) is a type of optimal packing of lines in Euclidean space.  A regular simplex is a special type of ETF in which the number of vectors is one more than the dimension of the space they span.  In this talk, we consider ETFs that contain a regular simplex, that is, have the property that a subset of its vectors forms a regular simplex.  As we explain, such ETFs are characterized as those that achieve equality in a certain well-known bound from the theory of compressed sensing.  We then consider the so-called binder of such an ETF, namely the set of all regular simplices that it contains.  In certain circumstances, we show this binder can be used to produce a particularly elegant Naimark complement of the corresponding ETF.  We also apply these ideas to known constructions of ETFs, including harmonic ETFs.


\setlength{\arraycolsep}{2pt}

\begin{document}

%%%%%%%%%%%%%%%%%%%%%%%%%%%%%%%%%%%%%%%%%%%%%%%%%%%%%%%%%%%%%%%%
\begin{frame}[plain,noframenumbering]
%%%%%%%%%%%%%%%%%%%%%%%%%%%%%%%%%%%%%%%%%%%%%%%%%%%%%%%%%%%%%%%%
\titlepage
\end{frame}

%%%%%%%%%%%%%%%%%%%%%%%%%%%%%%%%%%%%%%%%%%%%%%%%%%%%%%%%%%%%%%%%
\begin{frame}[noframenumbering]
%%%%%%%%%%%%%%%%%%%%%%%%%%%%%%%%%%%%%%%%%%%%%%%%%%%%%%%%%%%%%%%%
\title{Part I:\\
ETFs, ECTFFs, EITFFs, and an Open Conjecture}
\author{}
\institute{}

\date{}
\titlegraphic{}
\maketitle

\end{frame}

%%%%%%%%%%%%%%%%%%%%%%%%%%%%%%%%%%%%%%%%%%%%%%%%%%%%%%%%%%%%%%%%
\begin{frame}{Notation}
%%%%%%%%%%%%%%%%%%%%%%%%%%%%%%%%%%%%%%%%%%%%%%%%%%%%%%%%%%%%%%%%

Given $N$ vectors $\set{\bfphi_n}_{n\in\calN}$ in a $D$-dimensional Hilbert space $\bbH$:

\begin{itemize}

\vfill
\item
The \textbf{synthesis operator} is $\bfPhi:\bbF^\calN\rightarrow\bbH$,
\begin{equation*}
\bfPhi\bfx=\sum_{n\in\calN}\bfx(n)\bfphi_n.
\end{equation*}

\vfill
\item
The \textbf{frame operator} is $\bfPhi\bfPhi^*:\bbH\rightarrow\bbH$,
\begin{equation*}
\bfPhi\bfPhi^*\bfy
=\sum_{n\in\calN}\ip{\bfphi_n}{\bfy}\bfphi_n.
\end{equation*}

\vfill
\item
The \textbf{Gram matrix} is the $\calN\times\calN$ matrix,
\begin{equation*}
(\bfPhi^*\bfPhi)(n,n')
=\ip{\bfphi_n}{\bfphi_{n'}}.
\end{equation*}

\end{itemize}
\end{frame}

%%%%%%%%%%%%%%%%%%%%%%%%%%%%%%%%%%%%%%%%%%%%%%%%%%%%%%%%%%%%%%%%
\begin{frame}{Example: Tetrahedron, $D=3$, $N=4$}
%%%%%%%%%%%%%%%%%%%%%%%%%%%%%%%%%%%%%%%%%%%%%%%%%%%%%%%%%%%%%%%%

\begin{align*}
\bfPhi
&=\tfrac{1}{\sqrt{3}}\left[\begin{array}{rrrr}
 1&-1& 1&-1\\
 1& 1&-1&-1\\
 1&-1&-1& 1
\end{array}\right] \qquad 
\bfPhi^*=\left[\begin{array}{rrr}
 1& 1& 1\\
-1& 1&-1\\
 1&-1&-1\\
-1&-1& 1
\end{array}\right]\\\\
\bfPhi\bfPhi^*
&=\tfrac{1}{3}\left[\begin{array}{rrrr}
 1&-1& 1&-1\\
 1& 1&-1&-1\\
 1&-1&-1& 1
\end{array}\right]
\left[\begin{array}{rrr}
 1& 1& 1\\
-1& 1&-1\\
 1&-1&-1\\
-1&-1& 1
\end{array}\right]
=\left[\begin{array}{rrrr}
\tfrac{4}{3}&0&0\\
0&\tfrac{4}{3}&0\\
0&0&\tfrac{4}{3}
\end{array}\right]\\\\
\bfPhi^*\bfPhi
&=\tfrac{1}{3}\left[\begin{array}{rrr}
 1& 1& 1\\
-1& 1&-1\\
 1&-1&-1\\
-1&-1& 1
\end{array}\right]
\left[\begin{array}{rrrr}
 1&-1& 1&-1\\
 1& 1&-1&-1\\
 1&-1&-1& 1
\end{array}\right]
=\left[\begin{array}{rrrr}
 1&-\tfrac{1}{3}&-\tfrac{1}{3}&-\tfrac{1}{3}\\
-\tfrac{1}{3}& 1&-\tfrac{1}{3}&-\tfrac{1}{3}\\
-\tfrac{1}{3}&-\tfrac{1}{3}& 1&-\tfrac{1}{3}\\
-\tfrac{1}{3}&-\tfrac{1}{3}&-\tfrac{1}{3}& 1
\end{array}\right]
\end{align*}
\end{frame}

%%%%%%%%%%%%%%%%%%%%%%%%%%%%%%%%%%%%%%%%%%%%%%%%%%%%%%%%%%%%%%%%
\begin{frame}{Equiangular Tight Frame}
%%%%%%%%%%%%%%%%%%%%%%%%%%%%%%%%%%%%%%%%%%%%%%%%%%%%%%%%%%%%%%%%
\textbf{Theorem:} [Welch 74]
If $\{\bfphi_n\}_{n\in\calN}$ is a sequence of $N$ unit vectors in a $D$-dimensional Hilbert space $\bbH$ where $N\geq D$, then
\begin{equation*}
    \left[\tfrac{N-D}{D(N-1)}\right]^{\tfrac{1}{2}}\leq\max_{n\not=n'}\abs{\ip{\bfphi_n}{\bfphi_{n'}}}=\cos(\min_{n\not=n'}\theta_{n,n'}).
\end{equation*}

\vfill

\textbf{Definition:} $\{\bfphi_n\}_{n\in\calN}$ is an \textbf{equiangluar tight frame (ETF)} for $\bbH$ if both
\begin{enumerate}
    \item there exists $A>0$ such that $\bfPhi\bfPhi^*=A\bfI$ (tightness) and
    \item $\abs{\ip{\bfphi_n}{\bfphi_{n'}}}$ is constant over all $n\not=n'$ (equiangular).
\end{enumerate}
If $N=D+1$ an ETF for $\bbH$ is called a \textbf{regular $D$-simplex}.

\vfill

Equality is achieved in the Welch bound if and only if $\{\bfphi_n\}_{n\in\calN}$ is an ETF for $\bbH$.

\end{frame}

%%%%%%%%%%%%%%%%%%%%%%%%%%%%%%%%%%%%%%%%%%%%%%%%%%%%%%%%%%%%%%%%
\begin{frame}{Example: $D=6$, $N=16$}
%%%%%%%%%%%%%%%%%%%%%%%%%%%%%%%%%%%%%%%%%%%%%%%%%%%%%%%%%%%%%%%%
\footnotesize{
\begin{equation*}
\bfPhi=\frac1{\sqrt{6}}\left[\begin{array}{rrrrrrrrrrrrrrrr}
1&-1& 1&-1& 1&-1& 1&-1& 1&-1& 1&-1& 1&-1& 1&-1\\
1&-1& 1&-1& 1&-1& 1&-1&-1& 1&-1& 1&-1& 1&-1& 1\\
1& 1&-1&-1& 1& 1&-1&-1& 1& 1&-1&-1& 1& 1&-1&-1\\
1& 1&-1&-1&-1&-1& 1& 1& 1& 1&-1&-1&-1&-1& 1& 1\\
1&-1&-1& 1& 1&-1&-1& 1& 1&-1&-1& 1& 1&-1&-1& 1\\
1&-1&-1& 1&-1& 1& 1&-1&-1& 1& 1&-1& 1&-1&-1& 1
\end{array}\right]
\end{equation*}
}
\end{frame}

%%%%%%%%%%%%%%%%%%%%%%%%%%%%%%%%%%%%%%%%%%%%%%%%%%%%%%%%%%%%%%%%
\begin{frame}[noframenumbering]{Example: $D=6$, $N=16$}
%%%%%%%%%%%%%%%%%%%%%%%%%%%%%%%%%%%%%%%%%%%%%%%%%%%%%%%%%%%%%%%%
\begin{equation*}
\bfPhi\bfPhi^*=\left[\begin{array}{cccccc}
    \tfrac{16}{6}  &  0  &  0  &  0  &  0  &  0\\
     0  & \tfrac{16}{6}  &  0  &  0  &  0  &  0\\
     0  &  0  & \tfrac{16}{6}  &  0  &  0  &  0\\
     0  &  0  &  0  & \tfrac{16}{6}  &  0  &  0\\
     0  &  0  &  0  &  0  & \tfrac{16}{6}  &  0\\
     0  &  0  &  0  &  0  &  0  & \tfrac{16}{6}
\end{array}\right]
\end{equation*}

\end{frame}

%%%%%%%%%%%%%%%%%%%%%%%%%%%%%%%%%%%%%%%%%%%%%%%%%%%%%%%%%%%%%%%%
\begin{frame}[noframenumbering]{Example: $D=6$, $N=16$}
%%%%%%%%%%%%%%%%%%%%%%%%%%%%%%%%%%%%%%%%%%%%%%%%%%%%%%%%%%%%%%%%
\footnotesize{
\begin{equation*}
\bfPhi^*\bfPhi=\tfrac{1}{6}\left[\begin{array}{rrrrrrrrrrrrrrrr}
 6  & -2  & -2  & -2  &  2  & -2  &  2  & -2  &  2  &  2  & -2  & -2  &  2  & -2  & -2  &  2\\
-2  &  6  & -2  & -2  & -2  &  2  & -2  &  2  &  2  &  2  & -2  & -2  & -2  &  2  &  2  & -2\\
-2  & -2  &  6  & -2  &  2  & -2  &  2  & -2  & -2  & -2  &  2  &  2  & -2  &  2  &  2  & -2\\
-2  & -2  & -2  &  6  & -2  &  2  & -2  &  2  & -2  & -2  &  2  &  2  &  2  & -2  & -2  &  2\\
 2  & -2  &  2  & -2  &  6  & -2  & -2  & -2  &  2  & -2  & -2  &  2  &  2  &  2  & -2  & -2\\
-2  &  2  & -2  &  2  & -2  &  6  & -2  & -2  & -2  &  2  &  2  & -2  &  2  &  2  & -2  & -2\\
 2  & -2  &  2  & -2  & -2  & -2  &  6  & -2  & -2  &  2  &  2  & -2  & -2  & -2  &  2  &  2\\
-2  &  2  & -2  &  2  & -2  & -2  & -2  &  6  &  2  & -2  & -2  &  2  & -2  & -2  &  2  &  2\\
 2  &  2  & -2  & -2  &  2  & -2  & -2  &  2  &  6  & -2  & -2  & -2  &  2  & -2  &  2  & -2\\
 2  &  2  & -2  & -2  & -2  &  2  &  2  & -2  & -2  &  6  & -2  & -2  & -2  &  2  & -2  &  2\\
-2  & -2  &  2  &  2  & -2  &  2  &  2  & -2  & -2  & -2  &  6  & -2  &  2  & -2  &  2  & -2\\
-2  & -2  &  2  &  2  &  2  & -2  & -2  &  2  & -2  & -2  & -2  &  6  & -2  &  2  & -2  &  2\\
 2  & -2  & -2  &  2  &  2  &  2  & -2  & -2  &  2  & -2  &  2  & -2  &  6  & -2  & -2  & -2\\
-2  &  2  &  2  & -2  &  2  &  2  & -2  & -2  & -2  &  2  & -2  &  2  & -2  &  6  & -2  & -2\\
-2  &  2  &  2  & -2  & -2  & -2  &  2  &  2  &  2  & -2  &  2  & -2  & -2  & -2  &  6  & -2\\
 2  & -2  & -2  &  2  & -2  & -2  &  2  &  2  & -2  &  2  & -2  &  2  & -2  & -2  & -2  &  6
\end{array}\right]
\end{equation*}
}

\end{frame}

%%%%%%%%%%%%%%%%%%%%%%%%%%%%%%%%%%%%%%%%%%%%%%%%%%%%%%%%%%%%%%%%
\begin{frame}{Tight Fusion Frames}
%%%%%%%%%%%%%%%%%%%%%%%%%%%%%%%%%%%%%%%%%%%%%%%%%%%%%%%%%%%%%%%%
Recall $\{\bfphi_n\}_{n\in\calN}$ is a tight frame for $\bbH$ if $\exists A>0$ such that 
\begin{equation*}
    A\bfI
    =\bfPhi\bfPhi^*
    =\left[\begin{array}{ccc}
    \bfphi_1 & \cdots & \bfphi_N  
    \end{array}\right]
    \left[\begin{array}{c}
    \bfphi_1^*\\
    \vdots\\
    \bfphi_N^*
    \end{array}\right]
    =\sum_{n\in\calN}\bfphi_n^{}\bfphi_n^*.
\end{equation*}

\vfill

\textbf{Definition:} Let $\{\calU_n\}_{n\in\calN}$ be $M$-dimensional subspaces of $\bbH$ and $\bfE_n$ the synthesis operator of an ONB for $\calU_n$. $\{\calU_n\}$ is a \textbf{tight fusion frame (TFF)} for $\bbH$ if $\exists A>0$ such that 
\begin{equation*}
    A\bfI=\sum_{n\in\calN}\bfE_n^{}\bfE_n^*=\sum_{n\in\calN}\bfP_n.
\end{equation*}

\end{frame}

%%%%%%%%%%%%%%%%%%%%%%%%%%%%%%%%%%%%%%%%%%%%%%%%%%%%%%%%%%%%%%%%
\begin{frame}{ECTFFs and EITFFs}
%%%%%%%%%%%%%%%%%%%%%%%%%%%%%%%%%%%%%%%%%%%%%%%%%%%%%%%%%%%%%%%%
\textbf{Definition:} Let $\{\calU_n\}_{n\in\calN}$ be $M$-dimensional subspaces of $\bbH$ and $\bfPhi_n$ the synthesis operator of an ONB for $\calU_n$. $\{\calU_n\}$ is an

\begin{enumerate}
    \item \textbf{equi-chordal TFF (ECTFF)} for $\bbH$ if it is a TFF and $\norm{\bfE_n^*\bfE_{n'}^{}}_{\Fro}$ is constant over all $n\not=n'$, achieving equality in
    \begin{equation*}
        \left[\tfrac{M(MN-D)}{D(N-1)}\right]^{1/2}\leq\max_{n\not=n'}\norm{\bfE_n^*\bfE_{n'}^{}}_{\Fro}=\max_{n\not=n'}\ip{\bfP_n}{\bfP_{n'}}.
    \end{equation*}
    \item \textbf{equi-isoclinic TFF (EITFF)} for $\bbH$ if it is a ECTFF and $\bfE_n^*\bfE_{n'}^{}$ has constant singular values, achieving equality in
    \begin{equation*}
        \left[\tfrac{MN-D}{D(N-1)}\right]^{1/2}\leq\max_{n\not=n'}\norm{\bfE_n^*\bfE_{n'}^{}}_2.
    \end{equation*}
\end{enumerate}

\end{frame}

%%%%%%%%%%%%%%%%%%%%%%%%%%%%%%%%%%%%%%%%%%%%%%%%%%%%%%%%%%%%%%%%
\begin{frame}{Example: ECTFF and EITFF}
%%%%%%%%%%%%%%%%%%%%%%%%%%%%%%%%%%%%%%%%%%%%%%%%%%%%%%%%%%%%%%%%
\footnotesize{

The following are $\bfE_0$, $\bfE_1$, $\bfE_2$, and $\bfE_3$, respectively, and form an ECTFF for a 6 dimensional space consisting of 4 subspaces of dimension 3:
\begin{equation*}
    \left[\begin{array}{rrr}
    1 & 0 & 0\\
    0 & 0 & 0\\
    0 & 1 & 0\\
    0 & 0 & 0\\
    0 & 0 & 1\\
    0 & 0 & 0
    \end{array}\right],\
    \left[\begin{array}{rrr}
    1 & 0 & 0\\
    0 & 0 & 0\\
    0 & 0 & 0\\
    0 & 1 & 0\\
    0 & 0 & 0\\
    0 & 0 & 1
    \end{array}\right],\
    \left[\begin{array}{rrr}
    0 & 0 & 0\\
    1 & 0 & 0\\
    0 & 1 & 0\\
    0 & 0 & 0\\
    0 & 0 & 0\\
    0 & 0 & 1
    \end{array}\right],\
    \left[\begin{array}{rrr}
    0 & 0 & 0\\
    1 & 0 & 0\\
    0 & 0 & 0\\
    0 & 1 & 0\\
    0 & 0 & 1\\
    0 & 0 & 0
    \end{array}\right].
\end{equation*}

\vfill

For any $n,n'=0,1,2,3$, $n\not=n'$ the cross-Gram matrices $\bfE_n^*\bfE_{n'}^{}$ are one of the following:
\begin{equation*}
    \left[\begin{array}{rrr}
    1 & 0 & 0\\
    0 & 0 & 0\\
    0 & 0 & 0
    \end{array}\right],\
    \left[\begin{array}{rrr}
    0 & 0 & 0\\
    0 & 1 & 0\\
    0 & 0 & 0
    \end{array}\right],\
    \left[\begin{array}{rrr}
    0 & 0 & 0\\
    0 & 0 & 0\\
    0 & 0 & 1
    \end{array}\right].
\end{equation*}
}
\end{frame}

%%%%%%%%%%%%%%%%%%%%%%%%%%%%%%%%%%%%%%%%%%%%%%%%%%%%%%%%%%%%%%%%
\begin{frame}[noframenumbering]{Example: ECTFF and EITFF}
%%%%%%%%%%%%%%%%%%%%%%%%%%%%%%%%%%%%%%%%%%%%%%%%%%%%%%%%%%%%%%%%
\footnotesize{
The following are $\bfE_0$, $\bfE_1$, $\bfE_2$, and $\bfE_3$, respectively, and form an EITFF for a 6 dimensional space consisting of 4 subspaces of dimension 2:
\begin{equation*}
    \tfrac1{\sqrt{3}}\left[\begin{array}{rr}
    1 & 0\\
    1 & 0\\
    1 & 0\\
    0 & 1\\
    0 & 1\\
    0 & 1\\
    \end{array}\right],\
    \tfrac1{\sqrt{3}}\left[\begin{array}{rr}
    -1 &  0\\
     1 &  0\\
    -1 &  0\\
     0 & -1\\
     0 &  1\\
     0 & -1\\
    \end{array}\right],\
    \tfrac1{\sqrt{3}}\left[\begin{array}{rr}
     1 &  0\\
    -1 &  0\\
    -1 &  0\\
     0 &  1\\
     0 & -1\\
     0 & -1\\
    \end{array}\right],\
    \tfrac1{\sqrt{3}}\left[\begin{array}{rr}
    -1 &  0\\
    -1 &  0\\
     1 &  0\\
     0 & -1\\
     0 & -1\\
     0 &  1\\
    \end{array}\right].
\end{equation*}

\vfill

For any $n,n'=0,1,2,3$, $n\not=n'$, 
\begin{equation*}
    \bfE_n^*\bfE_{n'}^{}=
    \left[\begin{array}{rr}
    -1 &  0 \\
     0 & -1
    \end{array}\right].
\end{equation*}
}
\end{frame}

%%%%%%%%%%%%%%%%%%%%%%%%%%%%%%%%%%%%%%%%%%%%%%%%%%%%%%%%%%%%%%%%
\begin{frame}{An EITFF Construction and Motivation}

%%%%%%%%%%%%%%%%%%%%%%%%%%%%%%%%%%%%%%%%%%%%%%%%%%%%%%%%%%%%%%%%

\textbf{Theorem:}
[Lemmens \& Seidel 73]\smallskip

If $\{\bfdelta_m\}_{m\in\calM}$ is any ONB for an $M$-dimensional space $\bbH$, and for each $m\in\calM$, $\{\bfphi_{m,n}\}_{m\in\calM}$ is an ETF$(D,N)$ for $\bbK$, then $\{\calU_n\}_{n\in\calN}$, $\calU_n:=\Span\{\delta_m\otimes \varphi_{m,n}\}_{m\in\calM}$ is an EITFF for $\bbK\otimes\bbH$.

\vfill

\textbf{Example:} 
\begin{equation*}
\left[\begin{array}{cc}
1 & 0 \\
0 & 1
\end{array}\right]
\otimes \tfrac{1}{\sqrt{3}}\left[
\begin{array}{rrrr}
1 & -1 & 1 & -1 \\
1 & 1 & -1 & -1 \\
1 & -1 & -1 & 1 
\end{array}
\right]
\end{equation*}

\vfill

\textbf{Conjecture:} If $N$ spaces of dimension $M$ form an EITFF for a $D$-dimensional Hilbert space, then $M$ divides $D$ and $\exists$ an $N$-vector ETF for a Hilbert space of dimension $\tfrac{D}{M}$.

\end{frame}

%%%%%%%%%%%%%%%%%%%%%%%%%%%%%%%%%%%%%%%%%%%%%%%%%%%%%%%%%%%%%%%%
\begin{frame}[plain,noframenumbering]
%%%%%%%%%%%%%%%%%%%%%%%%%%%%%%%%%%%%%%%%%%%%%%%%%%%%%%%%%%%%%%%%
\title{Part II:\\
Harmonic ETFs Comprised of Regular Simplices}
\author{}
\institute{}

\date{}
\titlegraphic{}
\maketitle

\end{frame}

%%%%%%%%%%%%%%%%%%%%%%%%%%%%%%%%%%%%%%%%%%%%%%%%%%%%%%%%%%%%%%%%
\begin{frame}{Harmonic ETFs}
%%%%%%%%%%%%%%%%%%%%%%%%%%%%%%%%%%%%%%%%%%%%%%%%%%%%%%%%%%%%%%%%
\textbf{Definition:} A \textbf{character} on a finite abelian group $\calG$ is a homomorphism $\gamma:\calG\to\bbT=\{z\in\bbC:\abs{z}=1\}$. The set of all characters on $\calG$ is called the \textbf{(Pontryagin) dual}, denoted $\hat{\calG}$.

\vfill

The \textbf{character table} of $\calG$, $\bfF$, is the $\calG\times\hat{\calG}$-indexed table whose $(g,\gamma)$th entry is $\bfF(g,\gamma)=\gamma(g)$. $\bfF^*$ is the \textbf{DFT} on $\calG$.

\vfill

Restricting the rows of the character table of $\calG$ to any $\calD\subseteq\calG$ forms a tight frame for $\bbC^\calD$.
\end{frame}

%%%%%%%%%%%%%%%%%%%%%%%%%%%%%%%%%%%%%%%%%%%%%%%%%%%%%%%%%%%%%%%%
\begin{frame}{Example: Character Table on $\bbZ_{15}$}
%%%%%%%%%%%%%%%%%%%%%%%%%%%%%%%%%%%%%%%%%%%%%%%%%%%%%%%%%%%%%%%%
Let $\omega=e^{\tfrac{2\pi i}{15}}.$ The character table of $\bbZ_{15}$ is
\footnotesize{
\begin{equation*}
\left[
\begin{array}{ccccccccccccccc}
\omega^{ 0}  &   \omega^{ 0}  &   \omega^{ 0}  &   \omega^{ 0}  &   \omega^{ 0}  &   \omega^{ 0}  &   \omega^{ 0}  &   \omega^{ 0}  &   \omega^{ 0}  &   \omega^{ 0}  &   \omega^{ 0}  &   \omega^{ 0}  &   \omega^{ 0}  &   \omega^{ 0}  &   \omega^{ 0}\\
\omega^{ 0}  &   \omega^{ 1}  &   \omega^{ 2}  &   \omega^{ 3}  &   \omega^{ 4}  &   \omega^{ 5}  &   \omega^{ 6}  &   \omega^{ 7}  &   \omega^{ 8}  &   \omega^{ 9}  &   \omega^{10}  &   \omega^{11}  &   \omega^{12}  &   \omega^{13}  &   \omega^{14}\\
\omega^{ 0}  &   \omega^{ 2}  &   \omega^{ 4}  &   \omega^{ 6}  &   \omega^{ 8}  &   \omega^{10}  &   \omega^{12}  &   \omega^{14}  &   \omega^{ 1}  &   \omega^{ 3}  &   \omega^{ 5}  &   \omega^{ 7}  &   \omega^{ 9}  &   \omega^{11}  &   \omega^{13}\\
\omega^{ 0}  &   \omega^{ 3}  &   \omega^{ 6}  &   \omega^{ 9}  &   \omega^{12}  &   \omega^{ 0}  &   \omega^{ 3}  &   \omega^{ 6}  &   \omega^{ 9}  &   \omega^{12}  &   \omega^{ 0}  &   \omega^{ 3}  &   \omega^{ 6}  &   \omega^{ 9}  &   \omega^{12}\\
\omega^{ 0}  &   \omega^{ 4}  &   \omega^{ 8}  &   \omega^{12}  &   \omega^{ 1}  &   \omega^{ 5}  &   \omega^{ 9}  &   \omega^{13}  &   \omega^{ 2}  &   \omega^{ 6}  &   \omega^{10}  &   \omega^{14}  &   \omega^{ 3}  &   \omega^{ 7}  &   \omega^{11}\\
\omega^{ 0}  &   \omega^{ 5}  &   \omega^{10}  &   \omega^{ 0}  &   \omega^{ 5}  &   \omega^{10}  &   \omega^{ 0}  &   \omega^{ 5}  &   \omega^{10}  &   \omega^{ 0}  &   \omega^{ 5}  &   \omega^{10}  &   \omega^{ 0}  &   \omega^{ 5}  &   \omega^{10}\\
\omega^{ 0}  &   \omega^{ 6}  &   \omega^{12}  &   \omega^{ 3}  &   \omega^{ 9}  &   \omega^{ 0}  &   \omega^{ 6}  &   \omega^{12}  &   \omega^{ 3}  &   \omega^{ 9}  &   \omega^{ 0}  &   \omega^{ 6}  &   \omega^{12}  &   \omega^{ 3}  &   \omega^{ 9}\\
\omega^{ 0}  &   \omega^{ 7}  &   \omega^{14}  &   \omega^{ 6}  &   \omega^{13}  &   \omega^{ 5}  &   \omega^{12}  &   \omega^{ 4}  &   \omega^{11}  &   \omega^{ 3}  &   \omega^{10}  &   \omega^{ 2}  &   \omega^{ 9}  &   \omega^{ 1}  &   \omega^{ 8}\\
\omega^{ 0}  &   \omega^{ 8}  &   \omega^{ 1}  &   \omega^{ 9}  &   \omega^{ 2}  &   \omega^{10}  &   \omega^{ 3}  &   \omega^{11}  &   \omega^{ 4}  &   \omega^{12}  &   \omega^{ 5}  &   \omega^{13}  &   \omega^{ 6}  &   \omega^{14}  &   \omega^{ 7}\\
\omega^{ 0}  &   \omega^{ 9}  &   \omega^{ 3}  &   \omega^{12}  &   \omega^{ 6}  &   \omega^{ 0}  &   \omega^{ 9}  &   \omega^{ 3}  &   \omega^{12}  &   \omega^{ 6}  &   \omega^{ 0}  &   \omega^{ 9}  &   \omega^{ 3}  &   \omega^{12}  &   \omega^{ 6}\\
\omega^{ 0}  &   \omega^{10}  &   \omega^{ 5}  &   \omega^{ 0}  &   \omega^{10}  &   \omega^{ 5}  &   \omega^{ 0}  &   \omega^{10}  &   \omega^{ 5}  &   \omega^{ 0}  &   \omega^{10}  &   \omega^{ 5}  &   \omega^{ 0}  &   \omega^{10}  &   \omega^{ 5}\\
\omega^{ 0}  &   \omega^{11}  &   \omega^{ 7}  &   \omega^{ 3}  &   \omega^{14}  &   \omega^{10}  &   \omega^{ 6}  &   \omega^{ 2}  &   \omega^{13}  &   \omega^{ 9}  &   \omega^{ 5}  &   \omega^{ 1}  &   \omega^{12}  &   \omega^{ 8}  &   \omega^{ 4}\\
\omega^{ 0}  &   \omega^{12}  &   \omega^{ 9}  &   \omega^{ 6}  &   \omega^{ 3}  &   \omega^{ 0}  &   \omega^{12}  &   \omega^{ 9}  &   \omega^{ 6}  &   \omega^{ 3}  &   \omega^{ 0}  &   \omega^{12}  &   \omega^{ 9}  &   \omega^{ 6}  &   \omega^{ 3}\\
\omega^{ 0}  &   \omega^{13}  &   \omega^{11}  &   \omega^{ 9}  &   \omega^{ 7}  &   \omega^{ 5}  &   \omega^{ 3}  &   \omega^{ 1}  &   \omega^{14}  &   \omega^{12}  &   \omega^{10}  &   \omega^{ 8}  &   \omega^{ 6}  &   \omega^{ 4}  &   \omega^{ 2}\\
\omega^{ 0}  &   \omega^{14}  &   \omega^{13}  &   \omega^{12}  &   \omega^{11}  &   \omega^{10}  &   \omega^{ 9}  &   \omega^{ 8}  &   \omega^{ 7}  &   \omega^{ 6}  &   \omega^{ 5}  &   \omega^{ 4}  &   \omega^{ 3}  &   \omega^{ 2}  &   \omega^{ 1}
\end{array}
\right]
\end{equation*}
}

\end{frame}

%%%%%%%%%%%%%%%%%%%%%%%%%%%%%%%%%%%%%%%%%%%%%%%%%%%%%%%%%%%%%%%%
\begin{frame}[noframenumbering]{Example: Character Table on $\bbZ_{15}$}
%%%%%%%%%%%%%%%%%%%%%%%%%%%%%%%%%%%%%%%%%%%%%%%%%%%%%%%%%%%%%%%%
Let $\omega=e^{\tfrac{2\pi i}{15}}.$ The character table of $\bbZ_{15}$ is
\footnotesize{
\begin{equation*}
\left[
\begin{tabu}{ccccccccccccccc}
 \omega^{ 0}  &   \omega^{ 0}  &   \omega^{ 0}  &   \omega^{ 0}  &   \omega^{ 0}  &   \omega^{ 0}  &   \omega^{ 0}  &   \omega^{ 0}  &   \omega^{ 0}  &   \omega^{ 0}  &   \omega^{ 0}  &   \omega^{ 0}  &   \omega^{ 0}  &   \omega^{ 0}  &   \omega^{ 0}\\
 \omega^{ 0}  &   \omega^{ 1}  &   \omega^{ 2}  &   \omega^{ 3}  &   \omega^{ 4}  &   \omega^{ 5}  &   \omega^{ 6}  &   \omega^{ 7}  &   \omega^{ 8}  &   \omega^{ 9}  &   \omega^{10}  &   \omega^{11}  &   \omega^{12}  &   \omega^{13}  &   \omega^{14}\\
 \omega^{ 0}  &   \omega^{ 2}  &   \omega^{ 4}  &   \omega^{ 6}  &   \omega^{ 8}  &   \omega^{10}  &   \omega^{12}  &   \omega^{14}  &   \omega^{ 1}  &   \omega^{ 3}  &   \omega^{ 5}  &   \omega^{ 7}  &   \omega^{ 9}  &   \omega^{11}  &   \omega^{13}\\
 \omega^{ 0}  &   \omega^{ 3}  &   \omega^{ 6}  &   \omega^{ 9}  &   \omega^{12}  &   \omega^{ 0}  &   \omega^{ 3}  &   \omega^{ 6}  &   \omega^{ 9}  &   \omega^{12}  &   \omega^{ 0}  &   \omega^{ 3}  &   \omega^{ 6}  &   \omega^{ 9}  &   \omega^{12}\\
 \omega^{ 0}  &   \omega^{ 4}  &   \omega^{ 8}  &   \omega^{12}  &   \omega^{ 1}  &   \omega^{ 5}  &   \omega^{ 9}  &   \omega^{13}  &   \omega^{ 2}  &   \omega^{ 6}  &   \omega^{10}  &   \omega^{14}  &   \omega^{ 3}  &   \omega^{ 7}  &   \omega^{11}\\
 \omega^{ 0}  &   \omega^{ 5}  &   \omega^{10}  &   \omega^{ 0}  &   \omega^{ 5}  &   \omega^{10}  &   \omega^{ 0}  &   \omega^{ 5}  &   \omega^{10}  &   \omega^{ 0}  &   \omega^{ 5}  &   \omega^{10}  &   \omega^{ 0}  &   \omega^{ 5}  &   \omega^{10}\\
 \omega^{ 0}  &   \omega^{ 6}  &   \omega^{12}  &   \omega^{ 3}  &   \omega^{ 9}  &   \omega^{ 0}  &   \omega^{ 6}  &   \omega^{12}  &   \omega^{ 3}  &   \omega^{ 9}  &   \omega^{ 0}  &   \omega^{ 6}  &   \omega^{12}  &   \omega^{ 3}  &   \omega^{ 9}\\
 \omega^{ 0}  &   \omega^{ 7}  &   \omega^{14}  &   \omega^{ 6}  &   \omega^{13}  &   \omega^{ 5}  &   \omega^{12}  &   \omega^{ 4}  &   \omega^{11}  &   \omega^{ 3}  &   \omega^{10}  &   \omega^{ 2}  &   \omega^{ 9}  &   \omega^{ 1}  &   \omega^{ 8}\\
\rowfont{\color{lightgray}}
 \omega^{ 0}  &   \omega^{ 8}  &   \omega^{ 1}  &   \omega^{ 9}  &   \omega^{ 2}  &   \omega^{10}  &   \omega^{ 3}  &   \omega^{11}  &   \omega^{ 4}  &   \omega^{12}  &   \omega^{ 5}  &   \omega^{13}  &   \omega^{ 6}  &   \omega^{14}  &   \omega^{ 7}\\
\rowfont{\color{lightgray}}
 \omega^{ 0}  &   \omega^{ 9}  &   \omega^{ 3}  &   \omega^{12}  &   \omega^{ 6}  &   \omega^{ 0}  &   \omega^{ 9}  &   \omega^{ 3}  &   \omega^{12}  &   \omega^{ 6}  &   \omega^{ 0}  &   \omega^{ 9}  &   \omega^{ 3}  &   \omega^{12}  &   \omega^{ 6}\\
\rowfont{\color{lightgray}}
 \omega^{ 0}  &   \omega^{10}  &   \omega^{ 5}  &   \omega^{ 0}  &   \omega^{10}  &   \omega^{ 5}  &   \omega^{ 0}  &   \omega^{10}  &   \omega^{ 5}  &   \omega^{ 0}  &   \omega^{10}  &   \omega^{ 5}  &   \omega^{ 0}  &   \omega^{10}  &   \omega^{ 5}\\
\rowfont{\color{lightgray}}
 \omega^{ 0}  &   \omega^{11}  &   \omega^{ 7}  &   \omega^{ 3}  &   \omega^{14}  &   \omega^{10}  &   \omega^{ 6}  &   \omega^{ 2}  &   \omega^{13}  &   \omega^{ 9}  &   \omega^{ 5}  &   \omega^{ 1}  &   \omega^{12}  &   \omega^{ 8}  &   \omega^{ 4}\\
\rowfont{\color{lightgray}}
 \omega^{ 0}  &   \omega^{12}  &   \omega^{ 9}  &   \omega^{ 6}  &   \omega^{ 3}  &   \omega^{ 0}  &   \omega^{12}  &   \omega^{ 9}  &   \omega^{ 6}  &   \omega^{ 3}  &   \omega^{ 0}  &   \omega^{12}  &   \omega^{ 9}  &   \omega^{ 6}  &   \omega^{ 3}\\
\rowfont{\color{lightgray}}
 \omega^{ 0}  &   \omega^{13}  &   \omega^{11}  &   \omega^{ 9}  &   \omega^{ 7}  &   \omega^{ 5}  &   \omega^{ 3}  &   \omega^{ 1}  &   \omega^{14}  &   \omega^{12}  &   \omega^{10}  &   \omega^{ 8}  &   \omega^{ 6}  &   \omega^{ 4}  &   \omega^{ 2}\\
\rowfont{\color{lightgray}}
 \omega^{ 0}  &   \omega^{14}  &   \omega^{13}  &   \omega^{12}  &   \omega^{11}  &   \omega^{10}  &   \omega^{ 9}  &   \omega^{ 8}  &   \omega^{ 7}  &   \omega^{ 6}  &   \omega^{ 5}  &   \omega^{ 4}  &   \omega^{ 3}  &   \omega^{ 2}  &   \omega^{ 1}
\end{tabu}
\right]
\end{equation*}
}

\end{frame}

%%%%%%%%%%%%%%%%%%%%%%%%%%%%%%%%%%%%%%%%%%%%%%%%%%%%%%%%%%%%%%%%
\begin{frame}[noframenumbering]{Example: Character Table on $\bbZ_{15}$}
%%%%%%%%%%%%%%%%%%%%%%%%%%%%%%%%%%%%%%%%%%%%%%%%%%%%%%%%%%%%%%%%
Let $\omega=e^{\tfrac{2\pi i}{15}}.$ The character table of $\bbZ_{15}$ is
\footnotesize{
\begin{equation*}
\left[
\begin{tabu}{ccccccccccccccc}
\rowfont{\color{lightgray}}
 \omega^{ 0}  &   \omega^{ 0}  &   \omega^{ 0}  &   \omega^{ 0}  &   \omega^{ 0}  &   \omega^{ 0}  &   \omega^{ 0}  &   \omega^{ 0}  &   \omega^{ 0}  &   \omega^{ 0}  &   \omega^{ 0}  &   \omega^{ 0}  &   \omega^{ 0}  &   \omega^{ 0}  &   \omega^{ 0}\\
\rowfont{\color{lightgray}}
 \omega^{ 0}  &   \omega^{ 1}  &   \omega^{ 2}  &   \omega^{ 3}  &   \omega^{ 4}  &   \omega^{ 5}  &   \omega^{ 6}  &   \omega^{ 7}  &   \omega^{ 8}  &   \omega^{ 9}  &   \omega^{10}  &   \omega^{11}  &   \omega^{12}  &   \omega^{13}  &   \omega^{14}\\
 \omega^{ 0}  &   \omega^{ 2}  &   \omega^{ 4}  &   \omega^{ 6}  &   \omega^{ 8}  &   \omega^{10}  &   \omega^{12}  &   \omega^{14}  &   \omega^{ 1}  &   \omega^{ 3}  &   \omega^{ 5}  &   \omega^{ 7}  &   \omega^{ 9}  &   \omega^{11}  &   \omega^{13}\\
 \omega^{ 0}  &   \omega^{ 3}  &   \omega^{ 6}  &   \omega^{ 9}  &   \omega^{12}  &   \omega^{ 0}  &   \omega^{ 3}  &   \omega^{ 6}  &   \omega^{ 9}  &   \omega^{12}  &   \omega^{ 0}  &   \omega^{ 3}  &   \omega^{ 6}  &   \omega^{ 9}  &   \omega^{12}\\
\rowfont{\color{lightgray}} 
 \omega^{ 0}  &   \omega^{ 4}  &   \omega^{ 8}  &   \omega^{12}  &   \omega^{ 1}  &   \omega^{ 5}  &   \omega^{ 9}  &   \omega^{13}  &   \omega^{ 2}  &   \omega^{ 6}  &   \omega^{10}  &   \omega^{14}  &   \omega^{ 3}  &   \omega^{ 7}  &   \omega^{11}\\
\rowfont{\color{lightgray}} 
 \omega^{ 0}  &   \omega^{ 5}  &   \omega^{10}  &   \omega^{ 0}  &   \omega^{ 5}  &   \omega^{10}  &   \omega^{ 0}  &   \omega^{ 5}  &   \omega^{10}  &   \omega^{ 0}  &   \omega^{ 5}  &   \omega^{10}  &   \omega^{ 0}  &   \omega^{ 5}  &   \omega^{10}\\
 \omega^{ 0}  &   \omega^{ 6}  &   \omega^{12}  &   \omega^{ 3}  &   \omega^{ 9}  &   \omega^{ 0}  &   \omega^{ 6}  &   \omega^{12}  &   \omega^{ 3}  &   \omega^{ 9}  &   \omega^{ 0}  &   \omega^{ 6}  &   \omega^{12}  &   \omega^{ 3}  &   \omega^{ 9}\\
 \omega^{ 0}  &   \omega^{ 7}  &   \omega^{14}  &   \omega^{ 6}  &   \omega^{13}  &   \omega^{ 5}  &   \omega^{12}  &   \omega^{ 4}  &   \omega^{11}  &   \omega^{ 3}  &   \omega^{10}  &   \omega^{ 2}  &   \omega^{ 9}  &   \omega^{ 1}  &   \omega^{ 8}\\
\rowfont{\color{lightgray}}
 \omega^{ 0}  &   \omega^{ 8}  &   \omega^{ 1}  &   \omega^{ 9}  &   \omega^{ 2}  &   \omega^{10}  &   \omega^{ 3}  &   \omega^{11}  &   \omega^{ 4}  &   \omega^{12}  &   \omega^{ 5}  &   \omega^{13}  &   \omega^{ 6}  &   \omega^{14}  &   \omega^{ 7}\\
 \omega^{ 0}  &   \omega^{ 9}  &   \omega^{ 3}  &   \omega^{12}  &   \omega^{ 6}  &   \omega^{ 0}  &   \omega^{ 9}  &   \omega^{ 3}  &   \omega^{12}  &   \omega^{ 6}  &   \omega^{ 0}  &   \omega^{ 9}  &   \omega^{ 3}  &   \omega^{12}  &   \omega^{ 6}\\
\rowfont{\color{lightgray}}
 \omega^{ 0}  &   \omega^{10}  &   \omega^{ 5}  &   \omega^{ 0}  &   \omega^{10}  &   \omega^{ 5}  &   \omega^{ 0}  &   \omega^{10}  &   \omega^{ 5}  &   \omega^{ 0}  &   \omega^{10}  &   \omega^{ 5}  &   \omega^{ 0}  &   \omega^{10}  &   \omega^{ 5}\\
 \omega^{ 0}  &   \omega^{11}  &   \omega^{ 7}  &   \omega^{ 3}  &   \omega^{14}  &   \omega^{10}  &   \omega^{ 6}  &   \omega^{ 2}  &   \omega^{13}  &   \omega^{ 9}  &   \omega^{ 5}  &   \omega^{ 1}  &   \omega^{12}  &   \omega^{ 8}  &   \omega^{ 4}\\
 \omega^{ 0}  &   \omega^{12}  &   \omega^{ 9}  &   \omega^{ 6}  &   \omega^{ 3}  &   \omega^{ 0}  &   \omega^{12}  &   \omega^{ 9}  &   \omega^{ 6}  &   \omega^{ 3}  &   \omega^{ 0}  &   \omega^{12}  &   \omega^{ 9}  &   \omega^{ 6}  &   \omega^{ 3}\\
\rowfont{\color{lightgray}}
 \omega^{ 0}  &   \omega^{13}  &   \omega^{11}  &   \omega^{ 9}  &   \omega^{ 7}  &   \omega^{ 5}  &   \omega^{ 3}  &   \omega^{ 1}  &   \omega^{14}  &   \omega^{12}  &   \omega^{10}  &   \omega^{ 8}  &   \omega^{ 6}  &   \omega^{ 4}  &   \omega^{ 2}\\
 \omega^{ 0}  &   \omega^{14}  &   \omega^{13}  &   \omega^{12}  &   \omega^{11}  &   \omega^{10}  &   \omega^{ 9}  &   \omega^{ 8}  &   \omega^{ 7}  &   \omega^{ 6}  &   \omega^{ 5}  &   \omega^{ 4}  &   \omega^{ 3}  &   \omega^{ 2}  &   \omega^{ 1}
\end{tabu}
\right]
\end{equation*}
}

\end{frame}

%%%%%%%%%%%%%%%%%%%%%%%%%%%%%%%%%%%%%%%%%%%%%%%%%%%%%%%%%%%%%%%%
\begin{frame}[noframenumbering]{Example: Character Table on $\bbZ_{15}$}
%%%%%%%%%%%%%%%%%%%%%%%%%%%%%%%%%%%%%%%%%%%%%%%%%%%%%%%%%%%%%%%%
Let $\omega=e^{\tfrac{2\pi i}{15}}.$ The character table of $\bbZ_{15}$ is
\footnotesize{
\begin{equation*}
\left[
\begin{tabu}{ccccccccccccccc}
\rowfont{\color{lightgray}}
 \omega^{ 0}  &   \omega^{ 0}  &   \omega^{ 0}  &   \omega^{ 0}  &   \omega^{ 0}  &   \omega^{ 0}  &   \omega^{ 0}  &   \omega^{ 0}  &   \omega^{ 0}  &   \omega^{ 0}  &   \omega^{ 0}  &   \omega^{ 0}  &   \omega^{ 0}  &   \omega^{ 0}  &   \omega^{ 0}\\
\rowfont{\color{lightgray}}
 \omega^{ 0}  &   \omega^{ 1}  &   \omega^{ 2}  &   \omega^{ 3}  &   \omega^{ 4}  &   \omega^{ 5}  &   \omega^{ 6}  &   \omega^{ 7}  &   \omega^{ 8}  &   \omega^{ 9}  &   \omega^{10}  &   \omega^{11}  &   \omega^{12}  &   \omega^{13}  &   \omega^{14}\\
\rowfont{\color{lightgray}}  
 \omega^{ 0}  &   \omega^{ 2}  &   \omega^{ 4}  &   \omega^{ 6}  &   \omega^{ 8}  &   \omega^{10}  &   \omega^{12}  &   \omega^{14}  &   \omega^{ 1}  &   \omega^{ 3}  &   \omega^{ 5}  &   \omega^{ 7}  &   \omega^{ 9}  &   \omega^{11}  &   \omega^{13}\\
 \omega^{ 0}  &   \omega^{ 3}  &   \omega^{ 6}  &   \omega^{ 9}  &   \omega^{12}  &   \omega^{ 0}  &   \omega^{ 3}  &   \omega^{ 6}  &   \omega^{ 9}  &   \omega^{12}  &   \omega^{ 0}  &   \omega^{ 3}  &   \omega^{ 6}  &   \omega^{ 9}  &   \omega^{12}\\
\rowfont{\color{lightgray}} 
 \omega^{ 0}  &   \omega^{ 4}  &   \omega^{ 8}  &   \omega^{12}  &   \omega^{ 1}  &   \omega^{ 5}  &   \omega^{ 9}  &   \omega^{13}  &   \omega^{ 2}  &   \omega^{ 6}  &   \omega^{10}  &   \omega^{14}  &   \omega^{ 3}  &   \omega^{ 7}  &   \omega^{11}\\
\rowfont{\color{lightgray}}  
 \omega^{ 0}  &   \omega^{ 5}  &   \omega^{10}  &   \omega^{ 0}  &   \omega^{ 5}  &   \omega^{10}  &   \omega^{ 0}  &   \omega^{ 5}  &   \omega^{10}  &   \omega^{ 0}  &   \omega^{ 5}  &   \omega^{10}  &   \omega^{ 0}  &   \omega^{ 5}  &   \omega^{10}\\
 \omega^{ 0}  &   \omega^{ 6}  &   \omega^{12}  &   \omega^{ 3}  &   \omega^{ 9}  &   \omega^{ 0}  &   \omega^{ 6}  &   \omega^{12}  &   \omega^{ 3}  &   \omega^{ 9}  &   \omega^{ 0}  &   \omega^{ 6}  &   \omega^{12}  &   \omega^{ 3}  &   \omega^{ 9}\\
 \omega^{ 0}  &   \omega^{ 7}  &   \omega^{14}  &   \omega^{ 6}  &   \omega^{13}  &   \omega^{ 5}  &   \omega^{12}  &   \omega^{ 4}  &   \omega^{11}  &   \omega^{ 3}  &   \omega^{10}  &   \omega^{ 2}  &   \omega^{ 9}  &   \omega^{ 1}  &   \omega^{ 8}\\
\rowfont{\color{lightgray}}  
 \omega^{ 0}  &   \omega^{ 8}  &   \omega^{ 1}  &   \omega^{ 9}  &   \omega^{ 2}  &   \omega^{10}  &   \omega^{ 3}  &   \omega^{11}  &   \omega^{ 4}  &   \omega^{12}  &   \omega^{ 5}  &   \omega^{13}  &   \omega^{ 6}  &   \omega^{14}  &   \omega^{ 7}\\
 \omega^{ 0}  &   \omega^{ 9}  &   \omega^{ 3}  &   \omega^{12}  &   \omega^{ 6}  &   \omega^{ 0}  &   \omega^{ 9}  &   \omega^{ 3}  &   \omega^{12}  &   \omega^{ 6}  &   \omega^{ 0}  &   \omega^{ 9}  &   \omega^{ 3}  &   \omega^{12}  &   \omega^{ 6}\\
\rowfont{\color{lightgray}}  
 \omega^{ 0}  &   \omega^{10}  &   \omega^{ 5}  &   \omega^{ 0}  &   \omega^{10}  &   \omega^{ 5}  &   \omega^{ 0}  &   \omega^{10}  &   \omega^{ 5}  &   \omega^{ 0}  &   \omega^{10}  &   \omega^{ 5}  &   \omega^{ 0}  &   \omega^{10}  &   \omega^{ 5}\\
 \omega^{ 0}  &   \omega^{11}  &   \omega^{ 7}  &   \omega^{ 3}  &   \omega^{14}  &   \omega^{10}  &   \omega^{ 6}  &   \omega^{ 2}  &   \omega^{13}  &   \omega^{ 9}  &   \omega^{ 5}  &   \omega^{ 1}  &   \omega^{12}  &   \omega^{ 8}  &   \omega^{ 4}\\
 \omega^{ 0}  &   \omega^{12}  &   \omega^{ 9}  &   \omega^{ 6}  &   \omega^{ 3}  &   \omega^{ 0}  &   \omega^{12}  &   \omega^{ 9}  &   \omega^{ 6}  &   \omega^{ 3}  &   \omega^{ 0}  &   \omega^{12}  &   \omega^{ 9}  &   \omega^{ 6}  &   \omega^{ 3}\\
 \omega^{ 0}  &   \omega^{13}  &   \omega^{11}  &   \omega^{ 9}  &   \omega^{ 7}  &   \omega^{ 5}  &   \omega^{ 3}  &   \omega^{ 1}  &   \omega^{14}  &   \omega^{12}  &   \omega^{10}  &   \omega^{ 8}  &   \omega^{ 6}  &   \omega^{ 4}  &   \omega^{ 2}\\
 \omega^{ 0}  &   \omega^{14}  &   \omega^{13}  &   \omega^{12}  &   \omega^{11}  &   \omega^{10}  &   \omega^{ 9}  &   \omega^{ 8}  &   \omega^{ 7}  &   \omega^{ 6}  &   \omega^{ 5}  &   \omega^{ 4}  &   \omega^{ 3}  &   \omega^{ 2}  &   \omega^{ 1}
\end{tabu}
\right]
\end{equation*}
}

\end{frame}

%%%%%%%%%%%%%%%%%%%%%%%%%%%%%%%%%%%%%%%%%%%%%%%%%%%%%%%%%%%%%%%%
\begin{frame}{Harmonic ETFs}
%%%%%%%%%%%%%%%%%%%%%%%%%%%%%%%%%%%%%%%%%%%%%%%%%%%%%%%%%%%%%%%%
Restricting the rows of the character table of $\calG$ to any $\calD\subseteq\calG$ forms an ETF if and only if $\calD$ is a \textbf{difference set} for $\calG$, i.e $\#\{(d,d):g=d-d'\}$ is constant over all nonzero $g\in\calG$.

\vfill

\textbf{Example:}
$\{3,6,7,9,11,12,13,14\}$ is a difference set for $\bbZ_{15}$:

\footnotesize{
\begin{equation*}
\begin{array}{r|rrrrrrrr}
 -& 3& 6&7 & 9&11&12&13&14\\\hline
 3& 0&12&11& 9& 7& 6& 5& 4\\
 6& 3& 0&14&12&10& 9& 8& 7\\
 7& 4& 1& 0&13&11&10& 9& 8\\
 9& 6& 3& 2& 0&13&12&11&10\\
11& 8& 5& 4& 2& 0&14&13&12\\
12& 9& 6& 5& 3& 1& 0&14&13\\
13&10& 7& 6& 4& 2& 1& 0&14\\
14&11& 8& 7& 5& 3& 2& 1& 0
\end{array}
\end{equation*}
}

\end{frame}

%%%%%%%%%%%%%%%%%%%%%%%%%%%%%%%%%%%%%%%%%%%%%%%%%%%%%%%%%%%%%%%%
\begin{frame}[noframenumbering]{Harmonic ETFs}
%%%%%%%%%%%%%%%%%%%%%%%%%%%%%%%%%%%%%%%%%%%%%%%%%%%%%%%%%%%%%%%%
Restricting the rows of the character table of $\calG$ to any $\calD\subseteq\calG$ forms an ETF if and only if $\calD$ is a \textbf{difference set} for $\calG$, i.e $\#\{(d,d):g=d-d'\}$ is constant over all nonzero $g\in\calG$.

\vfill

\textbf{Example:}
$\{3,6,7,9,11,12,13,14\}$ is a difference set for $\bbZ_{15}$:

\footnotesize{
\begin{equation*}
\begin{array}{r|rrrrrrrr}
 -& 3& 6&7 & 9&11&12&13&14\\\hline
 3& 0&12&11& 9& 7& 6& 5& 4\\
 6& 3& 0&14&12&10& 9& 8& 7\\
 7& 4& \textcolor{red}{1} & 0&13&11&10& 9& 8\\
 9& 6& 3& 2& 0&13&12&11&10\\
11& 8& 5& 4& 2& 0&14&13&12\\
12& 9& 6& 5& 3& \textcolor{red}{1}& 0&14&13\\
13&10& 7& 6& 4& 2& \textcolor{red}{1}& 0&14\\
14&11& 8& 7& 5& 3& 2& \textcolor{red}{1}& 0
\end{array}
\end{equation*}
}

\end{frame}

%%%%%%%%%%%%%%%%%%%%%%%%%%%%%%%%%%%%%%%%%%%%%%%%%%%%%%%%%%%%%%%%
\begin{frame}[noframenumbering]{Harmonic ETFs}
%%%%%%%%%%%%%%%%%%%%%%%%%%%%%%%%%%%%%%%%%%%%%%%%%%%%%%%%%%%%%%%%
Restricting the rows of the character table of $\calG$ to any $\calD\subseteq\calG$ forms an ETF if and only if $\calD$ is a \textbf{difference set} for $\calG$, i.e $\#\{(d,d):g=d-d'\}$ is constant over all nonzero $g\in\calG$.

\vfill

\textbf{Example:}
$\{3,6,7,9,11,12,13,14\}$ is a difference set for $\bbZ_{15}$:

\footnotesize{
\begin{equation*}
\begin{array}{r|rrrrrrrr}
 -& 3& 6&7 & 9&11&12&13&14\\\hline
 3& 0& \textcolor{red}{12} &11& 9& 7& 6& 5& 4\\
 6& 3& 0&14&\textcolor{red}{12}&10& 9& 8& 7\\
 7& 4& 1& 0&13&11&10& 9& 8\\
 9& 6& 3& 2& 0&13&\textcolor{red}{12}&11&10\\
11& 8& 5& 4& 2& 0&14&13&\textcolor{red}{12}\\
12& 9& 6& 5& 3& 1& 0&14&13\\
13&10& 7& 6& 4& 2& 1& 0&14\\
14&11& 8& 7& 5& 3& 2& 1& 0
\end{array}
\end{equation*}
}

\end{frame}

%%%%%%%%%%%%%%%%%%%%%%%%%%%%%%%%%%%%%%%%%%%%%%%%%%%%%%%%%%%%%%%%
\begin{frame}[noframenumbering]{Harmonic ETFs}
%%%%%%%%%%%%%%%%%%%%%%%%%%%%%%%%%%%%%%%%%%%%%%%%%%%%%%%%%%%%%%%%

Restricting the rows of the character table of $\calG$ to any $\calD\subseteq\calG$ forms an ETF if and only if $\calD$ is a \textbf{difference set} for $\calG$, i.e $\#\{(d,d):g=d-d'\}$ is constant over all nonzero $g\in\calG$.

\vfill

\textbf{Example:}
$\{3,6,7,9,11,12,13,14\}$ is a difference set for $\bbZ_{15}$:

\footnotesize{
\begin{equation*}
\begin{array}{r|rrrrrrrr}
 -& 3& 6&7 & 9&11&12&13&14\\\hline
 3& 0&12&11& 9& \textcolor{red}{7}& 6& 5& 4\\
 6& 3& 0&14&12&10& 9& 8& \textcolor{red}{7}\\
 7& 4& 1& 0&13&11&10& 9& 8\\
 9& 6& 3& 2& 0&13&12&11&10\\
11& 8& 5& 4& 2& 0&14&13&12\\
12& 9& 6& 5& 3& 1& 0&14&13\\
13&10& \textcolor{red}{7}& 6& 4& 2& 1& 0&14\\
14&11& 8& \textcolor{red}{7}& 5& 3& 2& 1& 0
\end{array}
\end{equation*}
}

\end{frame}

%%%%%%%%%%%%%%%%%%%%%%%%%%%%%%%%%%%%%%%%%%%%%%%%%%%%%%%%%%%%%%%%
\begin{frame}{Example: ETF(8,15)}
%%%%%%%%%%%%%%%%%%%%%%%%%%%%%%%%%%%%%%%%%%%%%%%%%%%%%%%%%%%%%%%%
\footnotesize{
\begin{equation*}
\bfPhi=\tfrac{1}{\sqrt{8}}\left[
\begin{array}{ccccccccccccccc}
 \omega^{ 0}  &   \omega^{ 3}  &   \omega^{ 6}  &   \omega^{ 9}  &   \omega^{12}  &   \omega^{ 0}  &   \omega^{ 3}  &   \omega^{ 6}  &   \omega^{ 9}  &   \omega^{12}  &   \omega^{ 0}  &   \omega^{ 3}  &   \omega^{ 6}  &   \omega^{ 9}  &   \omega^{12}\\
 \omega^{ 0}  &   \omega^{ 6}  &   \omega^{12}  &   \omega^{ 3}  &   \omega^{ 9}  &   \omega^{ 0}  &   \omega^{ 6}  &   \omega^{12}  &   \omega^{ 3}  &   \omega^{ 9}  &   \omega^{ 0}  &   \omega^{ 6}  &   \omega^{12}  &   \omega^{ 3}  &   \omega^{ 9}\\
 \omega^{ 0}  &   \omega^{ 7}  &   \omega^{14}  &   \omega^{ 6}  &   \omega^{13}  &   \omega^{ 5}  &   \omega^{12}  &   \omega^{ 4}  &   \omega^{11}  &   \omega^{ 3}  &   \omega^{10}  &   \omega^{ 2}  &   \omega^{ 9}  &   \omega^{ 1}  &   \omega^{ 8}\\
 \omega^{ 0}  &   \omega^{ 9}  &   \omega^{ 3}  &   \omega^{12}  &   \omega^{ 6}  &   \omega^{ 0}  &   \omega^{ 9}  &   \omega^{ 3}  &   \omega^{12}  &   \omega^{ 6}  &   \omega^{ 0}  &   \omega^{ 9}  &   \omega^{ 3}  &   \omega^{12}  &   \omega^{ 6}\\
 \omega^{ 0}  &   \omega^{11}  &   \omega^{ 7}  &   \omega^{ 3}  &   \omega^{14}  &   \omega^{10}  &   \omega^{ 6}  &   \omega^{ 2}  &   \omega^{13}  &   \omega^{ 9}  &   \omega^{ 5}  &   \omega^{ 1}  &   \omega^{12}  &   \omega^{ 8}  &   \omega^{ 4}\\
 \omega^{ 0}  &   \omega^{12}  &   \omega^{ 9}  &   \omega^{ 6}  &   \omega^{ 3}  &   \omega^{ 0}  &   \omega^{12}  &   \omega^{ 9}  &   \omega^{ 6}  &   \omega^{ 3}  &   \omega^{ 0}  &   \omega^{12}  &   \omega^{ 9}  &   \omega^{ 6}  &   \omega^{ 3}\\
 \omega^{ 0}  &   \omega^{13}  &   \omega^{11}  &   \omega^{ 9}  &   \omega^{ 7}  &   \omega^{ 5}  &   \omega^{ 3}  &   \omega^{ 1}  &   \omega^{14}  &   \omega^{12}  &   \omega^{10}  &   \omega^{ 8}  &   \omega^{ 6}  &   \omega^{ 4}  &   \omega^{ 2}\\
 \omega^{ 0}  &   \omega^{14}  &   \omega^{13}  &   \omega^{12}  &   \omega^{11}  &   \omega^{10}  &   \omega^{ 9}  &   \omega^{ 8}  &   \omega^{ 7}  &   \omega^{ 6}  &   \omega^{ 5}  &   \omega^{ 4}  &   \omega^{ 3}  &   \omega^{ 2}  &   \omega^{ 1}
\end{array}
\right]
\end{equation*}
}

\end{frame}

%%%%%%%%%%%%%%%%%%%%%%%%%%%%%%%%%%%%%%%%%%%%%%%%%%%%%%%%%%%%%%%%
\begin{frame}{Example: ETF(8,15)}
%%%%%%%%%%%%%%%%%%%%%%%%%%%%%%%%%%%%%%%%%%%%%%%%%%%%%%%%%%%%%%%%
\footnotesize{
\begin{equation*}
\bfPhi=\tfrac{1}{\sqrt{8}}\left[
\begin{array}{{@{} >{\color{red}}c
>{\color{green}}c
>{\color{blue}}c
>{\color{red}}c
>{\color{green}}c
>{\color{blue}}c
>{\color{red}}c
>{\color{green}}c
>{\color{blue}}c
>{\color{red}}c
>{\color{green}}c
>{\color{blue}}c
>{\color{red}}c
>{\color{green}}c
>{\color{blue}}c
@{}}}
 \omega^{ 0}  &   \omega^{ 3}  &   \omega^{ 6}  &   \omega^{ 9}  &   \omega^{12}  &   \omega^{ 0}  &   \omega^{ 3}  &   \omega^{ 6}  &   \omega^{ 9}  &   \omega^{12}  &   \omega^{ 0}  &   \omega^{ 3}  &   \omega^{ 6}  &   \omega^{ 9}  &   \omega^{12}\\
 \omega^{ 0}  &   \omega^{ 6}  &   \omega^{12}  &   \omega^{ 3}  &   \omega^{ 9}  &   \omega^{ 0}  &   \omega^{ 6}  &   \omega^{12}  &   \omega^{ 3}  &   \omega^{ 9}  &   \omega^{ 0}  &   \omega^{ 6}  &   \omega^{12}  &   \omega^{ 3}  &   \omega^{ 9}\\
 \omega^{ 0}  &   \omega^{ 7}  &   \omega^{14}  &   \omega^{ 6}  &   \omega^{13}  &   \omega^{ 5}  &   \omega^{12}  &   \omega^{ 4}  &   \omega^{11}  &   \omega^{ 3}  &   \omega^{10}  &   \omega^{ 2}  &   \omega^{ 9}  &   \omega^{ 1}  &   \omega^{ 8}\\
 \omega^{ 0}  &   \omega^{ 9}  &   \omega^{ 3}  &   \omega^{12}  &   \omega^{ 6}  &   \omega^{ 0}  &   \omega^{ 9}  &   \omega^{ 3}  &   \omega^{12}  &   \omega^{ 6}  &   \omega^{ 0}  &   \omega^{ 9}  &   \omega^{ 3}  &   \omega^{12}  &   \omega^{ 6}\\
 \omega^{ 0}  &   \omega^{11}  &   \omega^{ 7}  &   \omega^{ 3}  &   \omega^{14}  &   \omega^{10}  &   \omega^{ 6}  &   \omega^{ 2}  &   \omega^{13}  &   \omega^{ 9}  &   \omega^{ 5}  &   \omega^{ 1}  &   \omega^{12}  &   \omega^{ 8}  &   \omega^{ 4}\\
 \omega^{ 0}  &   \omega^{12}  &   \omega^{ 9}  &   \omega^{ 6}  &   \omega^{ 3}  &   \omega^{ 0}  &   \omega^{12}  &   \omega^{ 9}  &   \omega^{ 6}  &   \omega^{ 3}  &   \omega^{ 0}  &   \omega^{12}  &   \omega^{ 9}  &   \omega^{ 6}  &   \omega^{ 3}\\
 \omega^{ 0}  &   \omega^{13}  &   \omega^{11}  &   \omega^{ 9}  &   \omega^{ 7}  &   \omega^{ 5}  &   \omega^{ 3}  &   \omega^{ 1}  &   \omega^{14}  &   \omega^{12}  &   \omega^{10}  &   \omega^{ 8}  &   \omega^{ 6}  &   \omega^{ 4}  &   \omega^{ 2}\\
 \omega^{ 0}  &   \omega^{14}  &   \omega^{13}  &   \omega^{12}  &   \omega^{11}  &   \omega^{10}  &   \omega^{ 9}  &   \omega^{ 8}  &   \omega^{ 7}  &   \omega^{ 6}  &   \omega^{ 5}  &   \omega^{ 4}  &   \omega^{ 3}  &   \omega^{ 2}  &   \omega^{ 1}
\end{array}
\right]
\end{equation*}
}

\end{frame}

%%%%%%%%%%%%%%%%%%%%%%%%%%%%%%%%%%%%%%%%%%%%%%%%%%%%%%%%%%%%%%%%
\begin{frame}{Harmonic ETFs Comprised of Simplices}
%%%%%%%%%%%%%%%%%%%%%%%%%%%%%%%%%%%%%%%%%%%%%%%%%%%%%%%%%%%%%%%%
\textbf{Theorem:}
[Fickus, Jasper, King \& Mixon 17]\smallskip

If $\calD$ is a $D$-element difference set in an abelian group $\calG$ of order $G$ that is disjoint from a subgroup $\calH$ of $\calG$ of order 
\begin{equation*}
H:=\#(\calH)=\frac{G}{S+1}
\quad\text{ where }S=\biggbracket{\frac{D(G-1)}{G-D}}^{\frac12}.
\end{equation*}
then the corresponding harmonic $\ETF(D,N)$ is a disjoint union of $H$ regular $S$-simplices indexed by the cosets of the annihilator, $\calH^\perp=\{\gamma\in\hat{\calG}:\gamma(h)=1, \forall h\in\calH\}$, of $\calH$.

\vfill

\textbf{Example:} For $\calD=\{3,6,7,9,11,12,13,14\}$ in $\calG=\bbZ_{15}$
\begin{equation*}
    G=15,\ D=8,\ S=\left[\tfrac{D(G-1)}{G-D}\right]^{\tfrac{1}{2}}=4,\ H=\tfrac{G}{S+1}=3.
\end{equation*}
Therefore, $\calH=\{0,5,10\}$ and $\calH^\perp=\{0,3,6,9,12\}$.
\end{frame}

%%%%%%%%%%%%%%%%%%%%%%%%%%%%%%%%%%%%%%%%%%%%%%%%%%%%%%%%%%%%%%%%
\begin{frame}[noframenumbering]{Example: ETF(8,15)}
%%%%%%%%%%%%%%%%%%%%%%%%%%%%%%%%%%%%%%%%%%%%%%%%%%%%%%
\footnotesize{
\begin{equation*}
\left[
\begin{tabu}{ccccccccccccccc}
 \textcolor{red}{\omega^{ 0}}  &   \omega^{ 0}  &   \omega^{ 0}  &   \textcolor{red}{\omega^{ 0}}  &   \omega^{ 0}  &   \omega^{ 0}  &   \textcolor{red}{\omega^{ 0}}  &   \omega^{ 0}  &   \omega^{ 0}  &   \textcolor{red}{\omega^{ 0}}  &   \omega^{ 0}  &   \omega^{ 0}  &   \textcolor{red}{\omega^{ 0}}  &   \omega^{ 0}  &   \omega^{ 0}\\
\rowfont{\color{lightgray}} 
 \omega^{ 0}  &   \omega^{ 1}  &   \omega^{ 2}  &   \omega^{ 3}  &   \omega^{ 4}  &   \omega^{ 5}  &   \omega^{ 6}  &   \omega^{ 7}  &   \omega^{ 8}  &   \omega^{ 9}  &   \omega^{10}  &   \omega^{11}  &   \omega^{12}  &   \omega^{13}  &   \omega^{14}\\
\rowfont{\color{lightgray}}
 \omega^{ 0}  &   \omega^{ 2}  &   \omega^{ 4}  &   \omega^{ 6}  &   \omega^{ 8}  &   \omega^{10}  &   \omega^{12}  &   \omega^{14}  &   \omega^{ 1}  &   \omega^{ 3}  &   \omega^{ 5}  &   \omega^{ 7}  &   \omega^{ 9}  &   \omega^{11}  &   \omega^{13}\\
\rowfont{\color{lightgray}}
 \omega^{ 0}  &   \omega^{ 3}  &   \omega^{ 6}  &   \omega^{ 9}  &   \omega^{12}  &   \omega^{ 0}  &   \omega^{ 3}  &   \omega^{ 6}  &   \omega^{ 9}  &   \omega^{12}  &   \omega^{ 0}  &   \omega^{ 3}  &   \omega^{ 6}  &   \omega^{ 9}  &   \omega^{12}\\
\rowfont{\color{lightgray}}
 \omega^{ 0}  &   \omega^{ 4}  &   \omega^{ 8}  &   \omega^{12}  &   \omega^{ 1}  &   \omega^{ 5}  &   \omega^{ 9}  &   \omega^{13}  &   \omega^{ 2}  &   \omega^{ 6}  &   \omega^{10}  &   \omega^{14}  &   \omega^{ 3}  &   \omega^{ 7}  &   \omega^{11}\\
 \textcolor{red}{\omega^{ 0}}  &   \omega^{ 5}  &   \omega^{10}  &   \textcolor{red}{\omega^{ 0}}  &   \omega^{ 5}  &   \omega^{10}  &   \textcolor{red}{\omega^{ 0}}  &   \omega^{ 5}  &   \omega^{10}  &   \textcolor{red}{\omega^{ 0}}  &   \omega^{ 5}  &   \omega^{10}  &   \textcolor{red}{\omega^{ 0}}  &   \omega^{ 5}  &   \omega^{10}\\
\rowfont{\color{lightgray}} 
 \omega^{ 0}  &   \omega^{ 6}  &   \omega^{12}  &   \omega^{ 3}  &   \omega^{ 9}  &   \omega^{ 0}  &   \omega^{ 6}  &   \omega^{12}  &   \omega^{ 3}  &   \omega^{ 9}  &   \omega^{ 0}  &   \omega^{ 6}  &   \omega^{12}  &   \omega^{ 3}  &   \omega^{ 9}\\
\rowfont{\color{lightgray}}
 \omega^{ 0}  &   \omega^{ 7}  &   \omega^{14}  &   \omega^{ 6}  &   \omega^{13}  &   \omega^{ 5}  &   \omega^{12}  &   \omega^{ 4}  &   \omega^{11}  &   \omega^{ 3}  &   \omega^{10}  &   \omega^{ 2}  &   \omega^{ 9}  &   \omega^{ 1}  &   \omega^{ 8}\\
\rowfont{\color{lightgray}}
 \omega^{ 0}  &   \omega^{ 8}  &   \omega^{ 1}  &   \omega^{ 9}  &   \omega^{ 2}  &   \omega^{10}  &   \omega^{ 3}  &   \omega^{11}  &   \omega^{ 4}  &   \omega^{12}  &   \omega^{ 5}  &   \omega^{13}  &   \omega^{ 6}  &   \omega^{14}  &   \omega^{ 7}\\
\rowfont{\color{lightgray}} 
 \omega^{ 0}  &   \omega^{ 9}  &   \omega^{ 3}  &   \omega^{12}  &   \omega^{ 6}  &   \omega^{ 0}  &   \omega^{ 9}  &   \omega^{ 3}  &   \omega^{12}  &   \omega^{ 6}  &   \omega^{ 0}  &   \omega^{ 9}  &   \omega^{ 3}  &   \omega^{12}  &   \omega^{ 6}\\
 \textcolor{red}{\omega^{ 0}}  &   \omega^{10}  &   \omega^{ 5}  &   \textcolor{red}{\omega^{ 0}}  &   \omega^{10}  &   \omega^{ 5}  &   \textcolor{red}{\omega^{ 0}}  &   \omega^{10}  &   \omega^{ 5}  &   \textcolor{red}{\omega^{ 0}}  &   \omega^{10}  &   \omega^{ 5}  &   \textcolor{red}{\omega^{ 0}}  &   \omega^{10}  &   \omega^{ 5}\\
\rowfont{\color{lightgray}}
 \omega^{ 0}  &   \omega^{11}  &   \omega^{ 7}  &   \omega^{ 3}  &   \omega^{14}  &   \omega^{10}  &   \omega^{ 6}  &   \omega^{ 2}  &   \omega^{13}  &   \omega^{ 9}  &   \omega^{ 5}  &   \omega^{ 1}  &   \omega^{12}  &   \omega^{ 8}  &   \omega^{ 4}\\
\rowfont{\color{lightgray}}
 \omega^{ 0}  &   \omega^{12}  &   \omega^{ 9}  &   \omega^{ 6}  &   \omega^{ 3}  &   \omega^{ 0}  &   \omega^{12}  &   \omega^{ 9}  &   \omega^{ 6}  &   \omega^{ 3}  &   \omega^{ 0}  &   \omega^{12}  &   \omega^{ 9}  &   \omega^{ 6}  &   \omega^{ 3}\\
\rowfont{\color{lightgray}}
 \omega^{ 0}  &   \omega^{13}  &   \omega^{11}  &   \omega^{ 9}  &   \omega^{ 7}  &   \omega^{ 5}  &   \omega^{ 3}  &   \omega^{ 1}  &   \omega^{14}  &   \omega^{12}  &   \omega^{10}  &   \omega^{ 8}  &   \omega^{ 6}  &   \omega^{ 4}  &   \omega^{ 2}\\
\rowfont{\color{lightgray}}
 \omega^{ 0}  &   \omega^{14}  &   \omega^{13}  &   \omega^{12}  &   \omega^{11}  &   \omega^{10}  &   \omega^{ 9}  &   \omega^{ 8}  &   \omega^{ 7}  &   \omega^{ 6}  &   \omega^{ 5}  &   \omega^{ 4}  &   \omega^{ 3}  &   \omega^{ 2}  &   \omega^{ 1}
\end{tabu}
\right]
\end{equation*}
}
\end{frame}

%%%%%%%%%%%%%%%%%%%%%%%%%%%%%%%%%%%%%%%%%%%%%%%%%%%%%%%%%%%%%%%%
\begin{frame}{Main Result 1}
%%%%%%%%%%%%%%%%%%%%%%%%%%%%%%%%%%%%%%%%%%%%%%%%%%%%%%%%%%%%%%%%

\textbf{Theorem:} [Fickus \& S 19]\\\smallskip

If $\calD$ is a $D$-element difference set for an abelian group $\calG$ of order $G$ and is disjoint from a subgroup $\calH$ of $\calG$, then $\#(\calH):=H\leq\tfrac{G}{S+1}$. Moreover, $H=\tfrac{G}{S+1}$ if and only if $\#(\calD_g):=\#[(\calD-g)\cap\calH]=\tfrac{D}{S}$, $\forall g\not\in\calH$.

\vfill

\textbf{Example:} For $\calD=\{6,11,7,12,13,3,9,14\}$,
\begin{align*}
    \calD_0=\calD_5=\calD_{10}&=\emptyset\\
    \calD_1=\calD_2=\calD_4=\calD_8&=\{5,10\}
\end{align*}

\textbf{Definition}
A $D$-element difference set $\calD$ in an abelian group $\calG$ of order $G$ is \textbf{fine} if it is disjoint from a subgroup $\calH$ of $\calG$ of order $H=\tfrac{G}{S+1}$.

\end{frame}

%%%%%%%%%%%%%%%%%%%%%%%%%%%%%%%%%%%%%%%%%%%%%%%%%%%%%%%%%%%%%%%%
\begin{frame}{Example: ETF(8,15)}
%%%%%%%%%%%%%%%%%%%%%%%%%%%%%%%%%%%%%%%%%%%%%%%%%%%%%%%%%%%%%%%%
The fine difference set $\{6,11,7,12,13,3\}$ in $\bbZ_{15}$, the ETF$(8,15)$ is comprised of $3$ regular $4$-simplices (\textcolor{red}{$\bfPhi_0$}, \textcolor{green}{$\bfPhi_1$}, \textcolor{blue}{$\bfPhi_2$}):

\vfill

\footnotesize{
\begin{equation*}
\bfPhi=\tfrac{1}{\sqrt{8}}\left[
\begin{array}{{@{} >{\color{red}}c
>{\color{green}}c
>{\color{blue}}c
>{\color{red}}c
>{\color{green}}c
>{\color{blue}}c
>{\color{red}}c
>{\color{green}}c
>{\color{blue}}c
>{\color{red}}c
>{\color{green}}c
>{\color{blue}}c
>{\color{red}}c
>{\color{green}}c
>{\color{blue}}c
@{}}}
\omega^{ 0}  &  \omega^{ 6}  &  \omega^{12}  &  \omega^{ 3}  &  \omega^{ 9}  &  \omega^{ 0}  &  \omega^{ 6}  &  \omega^{12}  &  \omega^{ 3}  &  \omega^{ 9}  &  \omega^{ 0}  &  \omega^{ 6}  &  \omega^{12}  &  \omega^{ 3}  &  \omega^{ 9}\\
\omega^{ 0}  &  \omega^{11}  &  \omega^{ 7}  &  \omega^{ 3}  &  \omega^{14}  &  \omega^{10}  &  \omega^{ 6}  &  \omega^{ 2}  &  \omega^{13}  &  \omega^{ 9}  &  \omega^{ 5}  &  \omega^{ 1}  &  \omega^{12}  &  \omega^{ 8}  &  \omega^{ 4}\\
\omega^{ 0}  &  \omega^{ 7}  &  \omega^{14}  &  \omega^{ 6}  &  \omega^{13}  &  \omega^{ 5}  &  \omega^{12}  &  \omega^{ 4}  &  \omega^{11}  &  \omega^{ 3}  &  \omega^{10}  &  \omega^{ 2}  &  \omega^{ 9}  &  \omega^{ 1}  &  \omega^{ 8}\\
\omega^{ 0}  &  \omega^{12}  &  \omega^{ 9}  &  \omega^{ 6}  &  \omega^{ 3}  &  \omega^{ 0}  &  \omega^{12}  &  \omega^{ 9}  &  \omega^{ 6}  &  \omega^{ 3}  &  \omega^{ 0}  &  \omega^{12}  &  \omega^{ 9}  &  \omega^{ 6}  &  \omega^{ 3}\\
\omega^{ 0}  &  \omega^{13}  &  \omega^{11}  &  \omega^{ 9}  &  \omega^{ 7}  &  \omega^{ 5}  &  \omega^{ 3}  &  \omega^{ 1}  &  \omega^{14}  &  \omega^{12}  &  \omega^{10}  &  \omega^{ 8}  &  \omega^{ 6}  &  \omega^{ 4}  &  \omega^{ 2}\\
\omega^{ 0}  &  \omega^{ 3}  &  \omega^{ 6}  &  \omega^{ 9}  &  \omega^{12}  &  \omega^{ 0}  &  \omega^{ 3}  &  \omega^{ 6}  &  \omega^{ 9}  &  \omega^{12}  &  \omega^{ 0}  &  \omega^{ 3}  &  \omega^{ 6}  &  \omega^{ 9}  &  \omega^{12}\\
\omega^{ 0}  &  \omega^{ 9}  &  \omega^{ 3}  &  \omega^{12}  &  \omega^{ 6}  &  \omega^{ 0}  &  \omega^{ 9}  &  \omega^{ 3}  &  \omega^{12}  &  \omega^{ 6}  &  \omega^{ 0}  &  \omega^{ 9}  &  \omega^{ 3}  &  \omega^{12}  &  \omega^{ 6}\\
\omega^{ 0}  &  \omega^{14}  &  \omega^{13}  &  \omega^{12}  &  \omega^{11}  &  \omega^{10}  &  \omega^{ 9}  &  \omega^{ 8}  &  \omega^{ 7}  &  \omega^{ 6}  &  \omega^{ 5}  &  \omega^{ 4}  &  \omega^{ 3}  &  \omega^{ 2}  &  \omega^{ 1}
\end{array}
\right]
\end{equation*}
}

\end{frame}

%%%%%%%%%%%%%%%%%%%%%%%%%%%%%%%%%%%%%%%%%%%%%%%%%%%%%%%%%%%%%%%%
\begin{frame}[noframenumbering]{Example: ETF(8,15)}
%%%%%%%%%%%%%%%%%%%%%%%%%%%%%%%%%%%%%%%%%%%%%%%%%%%%%%%%%%%%%%%%
The fine difference set $\{3,6,7,9,11,12,13,14\}$ in $\bbZ_{15}$, the ETF$(8,15)$ is comprised of $3$ regular $4$-simplices (\textcolor{red}{$\bfPhi_0$}, \textcolor{green}{$\bfPhi_1$}, \textcolor{blue}{$\bfPhi_2$}):

\vfill

\footnotesize{
\begin{equation*}
\bfPhi=\tfrac{1}{\sqrt{8}}\left[
\begin{array}{{@{} >{\color{red}}c
>{\color{red}}c
>{\color{red}}c
>{\color{red}}c
>{\color{red}}c
>{\color{green}}c
>{\color{green}}c
>{\color{green}}c
>{\color{green}}c
>{\color{green}}c
>{\color{blue}}c
>{\color{blue}}c
>{\color{blue}}c
>{\color{blue}}c
>{\color{blue}}c
@{}}}
\omega^{ 0}  &  \omega^{ 3}  &  \omega^{ 6}  &  \omega^{ 9}  &  \omega^{12}  &  \omega^{ 6}  &  \omega^{ 9}  &  \omega^{12}  &  \omega^{ 0}  &  \omega^{ 3}  &  \omega^{12}  &  \omega^{ 0}  &  \omega^{ 3}  &  \omega^{ 6}  &  \omega^{ 9}\\
\omega^{ 0}  &  \omega^{ 3}  &  \omega^{ 6}  &  \omega^{ 9}  &  \omega^{12}  &  \omega^{11}  &  \omega^{14}  &  \omega^{ 2}  &  \omega^{ 5}  &  \omega^{ 8}  &  \omega^{ 7}  &  \omega^{10}  &  \omega^{13}  &  \omega^{ 1}  &  \omega^{ 4}\\
\omega^{ 0}  &  \omega^{ 6}  &  \omega^{12}  &  \omega^{ 3}  &  \omega^{ 9}  &  \omega^{ 7}  &  \omega^{13}  &  \omega^{ 4}  &  \omega^{10}  &  \omega^{ 1}  &  \omega^{14}  &  \omega^{ 5}  &  \omega^{11}  &  \omega^{ 2}  &  \omega^{ 8}\\
\omega^{ 0}  &  \omega^{ 6}  &  \omega^{12}  &  \omega^{ 3}  &  \omega^{ 9}  &  \omega^{12}  &  \omega^{ 3}  &  \omega^{ 9}  &  \omega^{ 0}  &  \omega^{ 6}  &  \omega^{ 9}  &  \omega^{ 0}  &  \omega^{ 6}  &  \omega^{12}  &  \omega^{ 3}\\
\omega^{ 0}  &  \omega^{ 9}  &  \omega^{ 3}  &  \omega^{12}  &  \omega^{ 6}  &  \omega^{13}  &  \omega^{ 7}  &  \omega^{ 1}  &  \omega^{10}  &  \omega^{ 4}  &  \omega^{11}  &  \omega^{ 5}  &  \omega^{14}  &  \omega^{ 8}  &  \omega^{ 2}\\
\omega^{ 0}  &  \omega^{ 9}  &  \omega^{ 3}  &  \omega^{12}  &  \omega^{ 6}  &  \omega^{ 3}  &  \omega^{12}  &  \omega^{ 6}  &  \omega^{ 0}  &  \omega^{ 9}  &  \omega^{ 6}  &  \omega^{ 0}  &  \omega^{ 9}  &  \omega^{ 3}  &  \omega^{12}\\
\omega^{ 0}  &  \omega^{12}  &  \omega^{ 9}  &  \omega^{ 6}  &  \omega^{ 3}  &  \omega^{ 9}  &  \omega^{ 6}  &  \omega^{ 3}  &  \omega^{ 0}  &  \omega^{12}  &  \omega^{ 3}  &  \omega^{ 0}  &  \omega^{12}  &  \omega^{ 9}  &  \omega^{ 6}\\
\omega^{ 0}  &  \omega^{12}  &  \omega^{ 9}  &  \omega^{ 6}  &  \omega^{ 3}  &  \omega^{14}  &  \omega^{11}  &  \omega^{ 8}  &  \omega^{ 5}  &  \omega^{ 2}  &  \omega^{13}  &  \omega^{10}  &  \omega^{ 7}  &  \omega^{ 4}  &  \omega^{ 1}
\end{array}
\right]
\end{equation*}
}

\end{frame}

%%%%%%%%%%%%%%%%%%%%%%%%%%%%%%%%%%%%%%%%%%%%%%%%%%%%%%%%%%%%%%%%
\begin{frame}{Example: ETF($8,15$) Picking a Basis}
%%%%%%%%%%%%%%%%%%%%%%%%%%%%%%%%%%%%%%%%%%%%%%%%%%%%%%%%%%%%%%%%
\textbf{Example:} Each $4$-simplex in $8$-dimensions in the ETF$(8,15)$ is an embedding of the canonical 4-simplex in $4$-dimensions:
\begin{equation*}
\color{red}
\bfPhi_0
\color{black}
=
\color{red}
\tfrac{1}{\sqrt{2}}\left[\begin{array}{cccc}
	1 & 0 & 0 & 0\\
	1 & 0 & 0 & 0\\
	0 & 1 & 0 & 0\\
	0 & 1 & 0 & 0\\
	0 & 0 & 1 & 0\\
	0 & 0 & 1 & 0\\
	0 & 0 & 0 & 1\\
	0 & 0 & 0 & 1
	\end{array}\right]
	\color{black}
	\tfrac{1}{2}\left[\begin{array}{ccccc}
    1 & \omega^{3} &\omega^{6} &\omega^{9} & \omega^{12}\\
    1 & \omega^{6} &\omega^{12} &\omega^{3} & \omega^{9}\\
    1 & \omega^{9} &\omega^{3} &\omega^{12} & \omega^{6}\\
    1 & \omega^{12} &\omega^{9} &\omega^{6} & \omega^{3}\\
    \end{array}\right]
\end{equation*}


\end{frame}

%%%%%%%%%%%%%%%%%%%%%%%%%%%%%%%%%%%%%%%%%%%%%%%%%%%%%%%%%%%%%%%%
\begin{frame}[noframenumbering]{Example: ETF($8,15$) Picking a Basis}
%%%%%%%%%%%%%%%%%%%%%%%%%%%%%%%%%%%%%%%%%%%%%%%%%%%%%%%%%%%%%%%%
\textbf{Example:} Each $4$-simplex in $8$-dimensions in the ETF$(8,15)$ is an embedding of the canonical 4-simplex in $4$-dimensions:
\begin{equation*}
\color{green}
\bfPhi_1
\color{black}
=
\color{green}
\tfrac{1}{\sqrt{2}}\left[\begin{array}{cccc}
\omega^6 & 0 & 0 & 0\\
	\omega^{11} & 0 & 0 & 0\\
	0 & \omega^7 & 0 & 0\\
	0 & \omega^{12} & 0 & 0\\
	0 & 0 & \omega^{13} & 0\\
	0 & 0 & \omega^{3} & 0\\
	0 & 0 & 0 & \omega^9\\
	0 & 0 & 0 & \omega^{14}
	\end{array}\right]
	\color{black}
	\tfrac{1}{2}\left[\begin{array}{ccccc}
    1 & \omega^{3} &\omega^{6} &\omega^{9} & \omega^{12}\\
    1 & \omega^{6} &\omega^{12} &\omega^{3} & \omega^{9}\\
    1 & \omega^{9} &\omega^{3} &\omega^{12} & \omega^{6}\\
    1 & \omega^{12} &\omega^{9} &\omega^{6} & \omega^{3}\\
    \end{array}\right]
\end{equation*}


\end{frame}

%%%%%%%%%%%%%%%%%%%%%%%%%%%%%%%%%%%%%%%%%%%%%%%%%%%%%%%%%%%%%%%%
\begin{frame}[noframenumbering]{Example: ETF($8,15$) Picking a Basis}
%%%%%%%%%%%%%%%%%%%%%%%%%%%%%%%%%%%%%%%%%%%%%%%%%%%%%%%%%%%%%%%%
\textbf{Example:} Each $4$-simplex in $8$-dimensions in the ETF$(8,15)$ is an embedding of the canonical 4-simplex in $4$-dimensions:
\begin{equation*}
\color{blue}
\bfPhi_2
\color{black}
=
\color{blue}
\tfrac{1}{\sqrt{2}}\left[\begin{array}{cccc}
\omega^{12} & 0 & 0 & 0\\
	\omega^{7} & 0 & 0 & 0\\
	0 & \omega^{14} & 0 & 0\\
	0 & \omega^{9} & 0 & 0\\
	0 & 0 & \omega^{11} & 0\\
	0 & 0 & \omega^6 & 0\\
	0 & 0 & 0 & \omega^3\\
	0 & 0 & 0 & \omega^{13}
	\end{array}\right]
	\color{black}
	\tfrac{1}{2}\left[\begin{array}{ccccc}
    1 & \omega^{3} &\omega^{6} &\omega^{9} & \omega^{12}\\
    1 & \omega^{6} &\omega^{12} &\omega^{3} & \omega^{9}\\
    1 & \omega^{9} &\omega^{3} &\omega^{12} & \omega^{6}\\
    1 & \omega^{12} &\omega^{9} &\omega^{6} & \omega^{3}\\
    \end{array}\right]
\end{equation*}

\end{frame}

%%%%%%%%%%%%%%%%%%%%%%%%%%%%%%%%%%%%%%%%%%%%%%%%%%%%%%%%%%%%%%%%
\begin{frame}{Main Result 2}
%%%%%%%%%%%%%%%%%%%%%%%%%%%%%%%%%%%%%%%%%%%%%%%%%%%%%%%%%%%%%%%%

\textbf{Theorem:} [Fickus \& S 19]\\\smallskip

Let $\calD$ be a fine difference set in an abelian group $\calG$ and define,

\begin{align*}
	\bfPsi\in\bbC^{(\calG/\calH)\backslash\{\overline{0}\}\times\calH^\perp},&\qquad \bfPsi(\overline{g},\gamma)=\tfrac{1}{\sqrt{S}}\gamma(g)\\
	\bfPhi_\gamma\in\bbC^{\calD\times\calH^\perp},& \qquad \bfPhi_\gamma(d,\gamma')=\tfrac{1}{\sqrt{D}}(\gamma\gamma')(d)\\
	\bfE_\gamma\in\bbC^{\calD\times\calG/\calH\setminus\{\overline{0}\}},& \qquad \bfE_\gamma(d,\overline{g}):=\tfrac{\sqrt{S}}{\sqrt{D}}\left\{\begin{array}{cc}
	\gamma(d), & \overline{d}=\overline{g}, \\
	0, & \overline{d}\not=\overline{g}.
	\end{array}\right.
\end{align*}

Then $\bfE_{\gamma}$ is an isometry $\forall \gamma\in\hat{\calG}$ and $\bfPhi_{\gamma}=\bfE_{\gamma}\bfPsi$.
In particular if $\bfPhi$ is a harmonic ETF arising from a fine difference set, it is comprised of regular simplices.

\end{frame}

%%%%%%%%%%%%%%%%%%%%%%%%%%%%%%%%%%%%%%%%%%%%%%%%%%%%%%%%%%%%%%%%
\begin{frame}[noframenumbering]
%%%%%%%%%%%%%%%%%%%%%%%%%%%%%%%%%%%%%%%%%%%%%%%%%%%%%%%%%%%%%%%%
\title{Part III:\\
EITFF Conjecture Revisited}
\author{}
\institute{}

\date{}
\titlegraphic{}
\maketitle

\end{frame}

%%%%%%%%%%%%%%%%%%%%%%%%%%%%%%%%%%%%%%%%%%%%%%%%%%%%%%%%%%%%%%%%
\begin{frame}{Harmonic ETFs Comprised of Simplices}
%%%%%%%%%%%%%%%%%%%%%%%%%%%%%%%%%%%%%%%%%%%%%%%%%%%%%%%%%%%%%%%%

\textbf{Theorem:}
[Fickus, Jasper, King \& Mixon 17]\smallskip

If an ETF for $\bbH$ is comprised of regular simplices, then the subspaces of $\bbH$ spanned by those simplices form an ECTFF for $\bbH$.

\vfill 

\textbf{Note:} The proof of this theorem relied on using the projections onto these subspaces instead of an ONB for the subspaces making it hard to compute the singular values.

\end{frame}

%%%%%%%%%%%%%%%%%%%%%%%%%%%%%%%%%%%%%%%%%%%%%%%%%%%%%%%%%%%%%%%%
\begin{frame}{Main Result 3}
%%%%%%%%%%%%%%%%%%%%%%%%%%%%%%%%%%%%%%%%%%%%%%%%%%%%%%%%%%%%%%%%

\textbf{Theorem:} [Fickus \& S 19]\\\smallskip

If $\calD$ is a fine difference set in an abelian group $\calG$. Then
\begin{itemize}
    \item The cross-Gram matrix $\bfE_{\gamma}^*\bfE_{\gamma'}^{}$ is diagonal.
    \item The subspaces spanned by the simplices that comprise the ETF$(D,G)$ form an EITFF for $\bbC^\calD$ if and only if $\calD$ is an \textbf{amalgam}, that is $\forall g\in\calG,$ $\calD_g$ is a difference set for $\calH$.
\end{itemize}

\vfill

\textbf{Note:} These EITFF support the previous conjecture: there are $H$ subspace of $\bbC^{\calD}$ each of dimension $S$ that form an EITFF for $\bbC^{\calD}$ and $S$ divides $D$. Further, there exists an $ETF(\tfrac{D}{S},H)$ namely the harmonic ETF yielded by $\calD_g$ for all $g\not\in\calH.$

\vfill

\textbf{Example:} For $\calD=\{6,11,7,12,13,3,9,14\}$ in $\bbZ_{15}$ $\calD_g=\{5,10\}$ is a difference set for $\calH=\{0,5,10\}$.
\end{frame}

%%%%%%%%%%%%%%%%%%%%%%%%%%%%%%%%%%%%%%%%%%%%%%%%%%%%%%%%%%%%%%%%
\begin{frame}{Example: ETF$(8,15)$ Revisited}
%%%%%%%%%%%%%%%%%%%%%%%%%%%%%%%%%%%%%%%%%%%%%%%%%%%%%%%%%%%%%%%%
For $\calD=\{6,11,7,12,13,2,9,14\}$ in $\bbZ_{15}$, $\bfE_0$, $\bfE_1$, and $\bfE_2$ are

\begin{footnotesize}

\begin{equation*}
	\tfrac{1}{\sqrt{2}}\left[\begin{array}{cccc}
	1 & 0 & 0 & 0\\
	1 & 0 & 0 & 0\\
	0 & 1 & 0 & 0\\
	0 & 1 & 0 & 0\\
	0 & 0 & 1 & 0\\
	0 & 0 & 1 & 0\\
	0 & 0 & 0 & 1\\
	0 & 0 & 0 & 1
	\end{array}\right], \ \tfrac{1}{\sqrt{2}}\left[\begin{array}{cccc}
	\omega^6 & 0 & 0 & 0\\
	\omega^{11} & 0 & 0 & 0\\
	0 & \omega^7 & 0 & 0\\
	0 & \omega^{12} & 0 & 0\\
	0 & 0 & \omega^{13} & 0\\
	0 & 0 & \omega^{3} & 0\\
	0 & 0 & 0 & \omega^9\\
	0 & 0 & 0 & \omega^{14}
	\end{array}\right], \ \tfrac{1}{\sqrt{2}}\left[\begin{array}{cccc}
	\omega^{12} & 0 & 0 & 0\\
	\omega^{7} & 0 & 0 & 0\\
	0 & \omega^{14} & 0 & 0\\
	0 & \omega^{9} & 0 & 0\\
	0 & 0 & \omega^{11} & 0\\
	0 & 0 & \omega^6 & 0\\
	0 & 0 & 0 & \omega^3\\
	0 & 0 & 0 & \omega^{13}
	\end{array}\right].
\end{equation*}

and $\bfE_0^*\bfE_1^{}$, $\bfE_0^*\bfE_2^{}$, and $\bfE_1^*\bfE_2^{}$, respectively, are

\begin{equation*}
\label{eq: 8x15 cross Grams}
    -\tfrac{1}{2}\left[\begin{array}{cccc}
    \omega & 0 & 0 & 0  \\
    0 & \omega^{2} & 0 & 0  \\
    0 & 0 & \omega^{8} & 0  \\
    0 & 0 & 0 & \omega^{4}  \\
    \end{array}\right], 
    -\tfrac{1}{2}\left[\begin{array}{cccc}
    \omega^{2} & 0 & 0 & 0  \\
    0 & \omega^{4} & 0 & 0  \\
    0 & 0 & \omega & 0  \\
    0 & 0 & 0 & \omega^{8}  \\
    \end{array}\right],
    -\tfrac{1}{2}\left[\begin{array}{cccc}
    \omega & 0 & 0 & 0  \\
    0 & \omega^{2} & 0 & 0  \\
    0 & 0 & \omega^{8} & 0  \\
    0 & 0 & 0 & \omega^{4}  \\
    \end{array}\right].
\end{equation*}

\end{footnotesize}

\end{frame}

%%%%%%%%%%%%%%%%%%%%%%%%%%%%%%%%%%%%%%%%%%%%%%%%%%%%%%%%%%%%%%%%
\begin{frame}{Main Result 4}
%%%%%%%%%%%%%%%%%%%%%%%%%%%%%%%%%%%%%%%%%%%%%%%%%%%%%%%%%%%%%%%%
\textbf{Theorem:}

For any prime power $Q$, the complements of certain Singer difference set are amalgams and have parameters 
\begin{equation*}
    G=\tfrac{Q^J-1}{Q-1} \qquad D=Q^{J-1}
\end{equation*}
for any even integer $J\geq 4$.
The complements of twin prime power difference sets are amalgams when $Q\equiv 3\mod 4$ and have parameters
\begin{equation*}
    G=Q(Q+2) \qquad D=\tfrac{1}{2}(Q+1)^2
\end{equation*}
provided $Q+2$ is an odd prime power.
When $Q\equiv 1\mod 4$ these difference sets are not amalgams.
McFarland difference sets are never amalgams.

\end{frame}

%%%%%%%%%%%%%%%%%%%%%%%%%%%%%%%%%%%%%%%%%%%%%%%%%%%%%%%%%%%%%%%%
\begin{frame}{Main Result 5}
%%%%%%%%%%%%%%%%%%%%%%%%%%%%%%%%%%%%%%%%%%%%%%%%%%%%%%%%%%%%%%%%

\textbf{Definition:} A \textbf{conference matrix}, $\bfA$ is an $N\times N$ matrix whose diagonal entries are 0, off-diagonal entries have modulus 1, and satisfies that $\bfA\bfA^*=(N-1)\bfI$.

\vfill

\textbf{Theorem:}
Let $\calD$ be a fine difference set in a finite abelian group $\calG$ such that for all $g\in\calG$, $\calD_g$ is a difference set for $\calH$. For any $\gamma\in\hat{\calG}$, $\gamma\not\in\calH^\perp$ define 
\begin{equation*}
	\bfx\in\bbC^{\calG/\calH}, \qquad \bfx(\overline{g})=\tfrac{S^{3/2}}{D}\sum_{\substack{d\in\calD\\\overline{d}=\overline{g}}}\gamma(d).
\end{equation*}
Then the translates of $\bfx$ are orthogonal and 
$$\abs{\bfx(\overline{g})}=\left\{\begin{array}{cc}
0, & \overline{g}=\overline{0},\\
1, & \overline{g}\not=\overline{0}.
\end{array}\right.$$

\end{frame}

%%%%%%%%%%%%%%%%%%%%%%%%%%%%%%%%%%%%%%%%%%%%%%%%%%%%%%%%%%%%%%%%
\begin{frame}{Example: ETF$(8,15)$ Revisited}
%%%%%%%%%%%%%%%%%%%%%%%%%%%%%%%%%%%%%%%%%%%%%%%%%%%%%%%%%%%%%%%%
For $\calD=\{6,11,7,12,13,3,9,14\}$ in $\bbZ_{15}$,

\begin{equation*}
    \bfE_0^*\bfE_1^{}=-\tfrac{1}{2}\left[\begin{array}{cccc}
    \omega & 0 & 0 & 0  \\
    0 & \omega^{2} & 0 & 0  \\
    0 & 0 & \omega^{8} & 0  \\
    0 & 0 & 0 & \omega^{4}  \\
    \end{array}\right], 
\end{equation*}

and so the following is a complex circulant conference matrix satisfying $\bfA\bfA^*=4\bfI$:

\begin{equation*}
\bfA=-\left[
\begin{array}{ccccc}
0 & \omega^4 & \omega^8 & \omega^2 & \omega \\		
\omega & 0 & \omega^4 & \omega^8 & \omega^2 \\
\omega^2 & \omega & 0 & \omega^4 & \omega^8 \\
\omega^8 & \omega^2 & \omega & 0 & \omega^4 \\
\omega^4 & \omega^8 & \omega^2 & \omega & 0
\end{array}
\right]
\end{equation*}

\end{frame}

%%%%%%%%%%%%%%%%%%%%%%%%%%%%%%%%%%%%%%%%%%%%%%%%%%%%%%%%%%%%%%%%
\begin{frame}{Contributions}
%%%%%%%%%%%%%%%%%%%%%%%%%%%%%%%%%%%%%%%%%%%%%%%%%%%%%%%%%%%%%%%%
\begin{itemize}
    \item A fine difference set yields an EITFF if an only if it is an amalgam.
    
    \item Every amalgam yields a circulant conference matrix.
    
    \item There exists infinite family of fine difference sets that are amalgams and infinite families of fine difference sets that are not amalgams.
    
    \item Journal article.
    
    \item Conference proceeding for SampTA 2019.
\end{itemize}
\end{frame}

%%%%%%%%%%%%%%%%%%%%%%%%%%%%%%%%%%%%%%%%%%%%%%%%%%%%%%%%%%%%%%%%
\begin{frame}
%%%%%%%%%%%%%%%%%%%%%%%%%%%%%%%%%%%%%%%%%%%%%%%%%%%%%%%%%%%%%%%%
\begin{center}
    Thank you! Questions?
\end{center}

\end{frame}

\end{document}